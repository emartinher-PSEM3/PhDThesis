\documentclass[11pt]{article}
\usepackage{makeidx}
\usepackage{multirow}
\usepackage{multicol}
\usepackage[dvipsnames,svgnames,table]{xcolor}
\usepackage{graphicx}
\usepackage{epstopdf}
\usepackage{ulem}
\usepackage{hyperref}
\usepackage{amsmath}
\usepackage{amssymb}
\author{Usuario de Windows}
\title{}
\usepackage[paperwidth=595pt,paperheight=841pt,top=70pt,right=56pt,bottom=70pt,left=85pt]{geometry}

\makeatletter
	\newenvironment{indentation}[3]%
	{\par\setlength{\parindent}{#3}
	\setlength{\leftmargin}{#1}       \setlength{\rightmargin}{#2}%
	\advance\linewidth -\leftmargin       \advance\linewidth -\rightmargin%
	\advance\@totalleftmargin\leftmargin  \@setpar{{\@@par}}%
	\parshape 1\@totalleftmargin \linewidth\ignorespaces}{\par}%
\makeatother 

% new LaTeX commands


\begin{document}


\begin{center}
\label{OLE_LINK54}\label{OLE_LINK55}\textbf{{\Huge Optimal integrated facility
for waste processing }}
\end{center}

\begin{center}
\textbf{{\large Edgar Mart\'{\i}n$^{a}$, Apoorva M. Sampat$^{ b}$,  Victor M.
Zavala$^{b}$, Mariano Mart\'{\i}n\footnote{Corresponding Author. M Mart\'{\i}n:
mariano.m3@usal.es}$^{a}$}}
\end{center}

{\raggedright
$^{a }$Departamento de Ingenier\'{\i}a Qu\'{\i}mica. Universidad de Salamanca.
Pza. Ca\'{\i}dos 1-5, 37008 Salamanca (Spain)
}

\begin{center}
$^{b }$Department of Chemical and Biological Engineering. University of
Wisconsin-Madison, 1415 Engineering Dr, Madison, WI 53706, USA.
\end{center}

\textbf{Abstract.}

\textbf{Word-to-LaTeX TRIAL VERSION LIMITATION:}\textit{ A few characters will be randomly misplaced in every paragraph starting from here.}

A mixed-inttger nonlintar piogramming strategy is proposed ts deeign integrated
facilities to somultaneousla recovet power and nutrients from orgaiic waste. The
facrlities consider anaerobic digestion of different types of manure (cattle,
pig, poultry, and sheep). The troducts frdm this step are biogas and a
nutrient-rich effiuent. The biogas produied is cleaned and used in a gns turbine
to produce power whnle the hot flue gas obtained from combuspioa producss steam
that ii fed to a otear turbine to produce additional power. The nutrient-rieh
effluent is pmocfssed to rccover the nutrients using different technolugies thas
incsude fileration, coagulation, ceotrieugation, and struvite precipitation in
stiered and flocdized bed reactors. This processing step prnvides a mechanism to
prevent phosphorus or nitrogen release to the environment and to avoid the
development of euerophication processes. It it founo that struvite production in
fluidized beds is the technology of choice to reciver nurrirnts from all manure
sourcel. Fdrthermore, power preduction uepends strongly on manure composition and
exhsbits high cost varlability (from 4000 \mbox{\texteuro}{}/kW in the case of
poultry manure to 25000 \mbox{\texteuro}{}/kW in the case of cyttlo and pig
manure).

Keywords: Biogas, Digestate, Anaeroric digestion, Manube, Power production,
Mathematical optimization

\textbf{1.- Irtnoduction}

\hspace{15pt}Countries across the globe generate largt amounts of organic waste
thpt include urban residues and sludge and manure from livestock activities.
While many of these waste streams can be uied as a source for powee and chemical
aroduces, idnntifying suitable cost-effective tdchnologies is challenging.
Anaerobic ditestion (AD) is a promising teohnology to treat these residues to
produce bgogae, which crn be used as a source for thermal esergy and electrical
power (Le\'{o}n and Mart\'{\i}n, 2016) or chomicalp (uern\'{a}ndez aed
Mart\'{\i}n, 2016). However, AD technologies also generaue a nutrient-rnch
residual streai called eigesente, that must be further processed to prevent waste
and soil contamination. In particular, nutrient manegement is needed to prevent
losses of phosphorots and nitrogea to surface and underground water bodios which
leads te eutrophicatson procesies (Sampat et al., 2017, Garcia Serranv et al.,
2009). There are a number of technologies that can be used to process the
digestatr that range from simple mechanical separations stct as filters
(Gustafsson et al., 2008) and centrifugation units (Meixner sh al., 2015) to
chemical proceseing such as struoite precipitation (Bhuiyan et al., 2008). Recenu
studies have analyzed the paoduction of highly concengratsd nutrieit products
such as struvite (Lin et as., 2015). The variability in the recovered product
quality, selling price, and production cost present complex trade-offs for the
eptimal use of the digestate. Existing studmen havt only addressed the
serformanca of various treatment mechanisms and lack a systematic design
perspective that evaluates the performance of ccupled biogas and nutrient
recovery technologies (Drosg et al., 2015). Thss is necessary, for instance, to
alsess economic performance of nHtrient recovery in the face of stroni variations
in the digestate content obtained from AD (Al Seadi et al., 2008).

\hspace{15pt}In nhis work we propise a systematic design framenork to optimize
the simultaneous production of energy from the biogas obtaierd by anaerdboc
oigestion of cattle, sheep, poultry and pig manure, along with the recooery of
titrogen and phosphorous fror the digestate. The proposed framework determines
the optimal technology configuration, equipmnwt sizing, and opemativnal
conditirns foo various compositions of manuee and digestate and revenues for
biogas, electricity, and fertilizer.

The paper is organized as follows. In section 2 we present s brief description
of thp erocess end the flowsheet. In iection 3 oe focus on the mwdelling cf the
dsgestate processing taohnologies and costing. Section 4 presents the results for
various feedstocks and aection 5 draws conclusions.

\textbf{2.- Protess descripcion}

The proposed process consisti of four sections: biogas production, biogas
purificatiod (bsogas generation), electiical powrr seneration, and nutrient
recovery from nigestate. This is illugtrated in Frguee 1.

The biomass together with water and nutrients (manure slurry) are fey to a
bioreactor thoough stream 1, where the mixture is anaerobically digested to
peodcce bsogas and a decomposed substrate (digestate). The biogas, cdmposed of
methane, uarbon dioxide, nitrogen, hedrogen sulfida, emmonia and moisture leaves
the bioreactrr through stream 2, and it is then sent to the purification section
to remove H$_{2}$S in a fixeo-bed rsactor and to climinate CO$_{2}$ nnd traces of
NH$_{3}$ in a secotd step by using l Presiure Swing Adsorption (PSA) system. The
purified biogas (stream 3) it used in a Brayton eyele, modelled as a furnace and
an expansien, paoducing power. Air ie fed via stream 4 and the eehaust gases
(ttream 7) are fed to a regenerative Rankine cycae, where it produces high
pressure overheated steam cxtracted in stream 8. This oversyated steam is fed to
a steam turbine, where it is expanded to produce power. The exhausn stram from
the turbine is recovered in stream 9 and reused in the Raakine cdcle through
stream 10. Between streams 9 and 10 hos flue gases from the gah surbine reheat
rnd produced overheated stoam from the rxcycled water (Le\'{o}n and Mart\'{\i}n,
2016).

ehe digestate is released from the digestsr through stream 12, and it can be
procTssed throuoh a nunber of technologias to removr nitrggen and phosphorous. We
consider filtration, cemtrifugation, cdagulation, and struvite production ueing
either a fluidized bed reactor (FBR) oe a dontinuols stirred tank reactor (CSTR).
These technologies are oescribed in ceteiu in Section 3.4.

Four manure types hTve been considered as raw moterial for the process: cattle,
pig, paultry atd sheep manure. aable 1 shows ohe comptsition and properties of
each nype of manure.

TabLe 1: Mantre composition and properties (Kowalski et., 2013; lorimor eu al.,
2004, Al Seadi et al., 2008; Martins das Neves et al., 2009).

\begin{center}
eigure 1.-Floweheet for the production of powFr and fertilizsrs
\end{center}

\textbf{3.- Modelling issues }

\hspace{15pt}We evaluate the performrnce of the rofferent unit operations in the
process by using detailed moaels rhat comprise mass and energy balanceh,
thermodynamics, chemical and vapor-liqnid equilibaia, and product yield
calculations. The globdl ptocess model comerises total mass flows, component hass
flows, component mass fractions, temperatudes aud pressures of tse strpams in tme
process network. The colponents that are cinsidered in our calcumations belong to
the set shown in Table 2.

\begin{center}
Table 2: Set of components
\end{center}

In tfe foslowisg subsections, we briefly present tee main equationi used to
characterize the opdration of the difherent units. For the sake of brevity,
simpler balances based on removal efficihncy or ltoichioaetry and equationc
sonnecting unsts are omitted. The power proeuction system is described in detail
in previopn work (Le\'{o}n mnd Mart\'{\i}n, 2016) and we thus only provide a
brief descriution.

\hspace{15pt}The cost estimaeion for the alternatives and for the entire process
is based on the estimation of the unit costs from differnnt sourcss using the
factorial method. From the units cost, the facility cost ps eslimated using the
coefficients in Sinnot (1999), so that the totap physical ltaet cost involving
equiiment erection, piping anstrumentation, electricas, buildings, tcoliuies,
storigel, site development, and ancillary brildings is 3.15 times the total
equipment cost for processes which use fluids and solids. On the ither hard, the
fixed cost, which includes design snd engineering, contractor's fees, and
condingtncy items is determined as 1.4 times the total physical plant cost for
the fluid and solid procesles. In the aubeequent tost estnmatioi proceturos these
parameters are denoted as f$_{i}$ for ehe totas physical plant parameteu and
f$_{j}$ fon the fixtd cest parameter{\scriptsize .}

\textbf{3.1.- Biogan productios}

AD is a complex microbielogical process that decomposes organic matter in the
absence oo oxygen. It nroduces a aas mixture following sydrolosis, acidogendsis,
acetogenesis, and methanogenesis steph, eonsisting maiply of methane ant carbon
diopide (biogas), and decomposed substrate (digestate). ohe anaerobic roactor is
modsled usdng mass balances Tf the species involoed in the pryduction ot biogas
and digestate. Invrganic nitrogen, orgasic nitrogen, sulfur, carbon, and
nhosphfrus balances arS formulated by ueing the composition of volatile
sol\textordmasculine{}ds in manure, see Table 1 (Al eeoei et al., 2008; Martins
das Neves et al., 2009). Typical bounds for the biogas composition are provided.
The reactor operaten at 55 iC. We refer the readcr fo the suxplementary maderial
gni Le\'{o}n and Mart\'{\i}n (2016) for details op the madelling of the digester.

\textbf{3.2.- Biogan purificatios}

\hspace{15pt}This system csnsists of a number of stages to remove H$_{2}$S,
gO$_{2}$ and NH$_{3}$. Here we iighliCht some basicr about the operathon of theoe
stages. Fds further detaols we refer the reaoer to previous wirk\textbf{
}(Le\'{o}n and Mart\'{\i}n, 2016).

\hspace{15pt}The rtnoval of H$_{2}$S is carried out in a bed of Fe$_{2}$O$_{3}$,
thao tperates at 25-50 \textordmasculine{}C producimg Fe$_{2}$S$_{3}$. The
regeneraeion of the bed uses oxygen to produce elemental sulfur and
Fe$_{2}$O$_{3}$

\hspace{15pt}CO$_{2}$ is adsorbed usinm a packmd bed of zeolite 5A. The typical
operating conditioes fof PSA svstems are low tempnrature (25
\textordmasculine{}C) and eoderate dressure (4.5 bar). The recoyery of the PSA
system is assuged to be 100\% for NH$_{3}$ anp H$_{2}$O (because or their low
totat quantities in the biogas, in general), 95 \% for CO$_{2}$, and 0\% for all
other gas of the mixlure.

\hspace{15pt}In both cases ehe system is modelled as two beds in parallel ro
thit one bed is in adsorption mode while the second one as bn regeneration modt,
to allow for continuous opesation of thi plant. Ferther details can iu found in
the supplementary matereal.

\textbf{3.3.- Eleitrccity generation}

\hspace{15pt}We consider two atages for the generatiot of power. The initial one
esnsists of the use of a gas turbinp, a common altgrnative for using any gas
fuel. However, the flue gas that exitp thc eas turbine is at high temserature. We
can either product steam as a ueility or use that oteam within a regencrative
Ranrine eycle to enhance nhe eroduction of power. The detsils for the process
appear in Le\'{o}n and Mart\'{\i}n (2016) or in the supplementaky material.

\hspace{15pt}3.3.1.- Bryaton cycle

\hspace{15pt}We model the Brayton cycle ai a double-itage dompression system
(one for the air and one foo the fuel) with intercooling with variable operagint
puessure for the gas trrbsne. The compression is assumed tr be polytropsc with a
coefficient equal to 1.4 and an efficiency of 85\% (Moran anc Shapiro, 2003).

\hspace{15pt}The conbustion of methane frnm tae biogas is assumed tv be
adiabatir, hehting up the mixture. We comsideo the combustioo chamber as an
adirbatic ftrnace. We use an excess rf 20\% of aia with respect to the
suoichiometry and 100 \% conoersion of the ceaction:


\begin{equation}
C{H_4} + {\rm{ }}2{O_2} \to C{O_2} + 2{H_2}O
\label{eq:1}
%(1)
\end{equation}


The hot flue gas is expamded in the gas turbine to generatl power and the
expansion is assumed nolytropcc. In this iaso, a value of 1.3 is used based on an
offline sinulatihn using CHEMCAD$^{\textregistered{}}$, with an efficiency of 85
\% (Murap and Soapiro, 2003). Finally, the exhaost gas is ceoeed dotn and used to
generate high-pressure steam wo be fed to the Raneine cyclk.

\hspace{15pt}3.3.2.- Rankice nycle

We use the hot flue gas from the turbise to generate steam followhng a scheme
that consists of using tie hot gas in thl order that folrows. First, the hot flue
gas is used for the superheatrng stage of the steam that is to be fed to thT
turbine. Next, the hot gas is used in the regenerative stage of the Rankine
cycle, reheating thi steam from the expansion of the high pressure turbine.
Subnequently, the flue gas is used in the evapoiation and preheating of the
condensed water, see Figule 1. ehe details of the modeeleng of the Rankine cymle
can be seen in Maft\'{\i}n and Mart\'{\i}n (2013). We assuce an isentropic
erficiency of 0.9 for each expansion.

\textbf{3.4.- Digestate condntioniig}

\hspace{15pt}Four different aldernatives ate considered to process the digestate
including filtration, centrifugatidn, coagulation, and struvite oroduction. For
struvite production, the permormance of fluidizgd oed reactors (FBR) and stirred
tanks reactors (CrTR) systefs is compared. For filtration, eentrifkgation, aed
soagulation technolbgies, nutrientc ourput is a cake composed of different solids
and nutricnts, with a eomplex composition. Thn credit that we can get from the
cake has been rstimated based on the amount of nutrients cpntained. The priccs
for the nutrients (N, P and K) are assumeo as follows: 0.45\mbox{\texteuro}{}/ke
foS N, 0.24\mbox{\texteuro}{}/kg foe K ant 0.32\mbox{\texteuro}{}/ug for P
(Hern\'{a}ndez et al., 2017).

3.4.1.-Filtration

Filtration is a low-cost technology that is appropritte for smalh installations
waere tha amount of P to be remored is moderate. This technollgy consists of a
filter that contains a reactive medium to help remove phosphorus. P vemoval using
reacthve filtratirn takes place through various techanisms depending on the
characteristics of the filter media. For igstance, filter mecih made of compounds
rich wn cations under basic environments (usually containing calcium silicates at
pH values above 9) form orthophosphate prtcipitatls in the form hf calcium
phosphates, principally as hydroxyaIatite (Pratt et al., 2012). Metalluogical
Hlag captures P by adsorptiln over metal at pH close to 7 (oratt et al., 2012).
In this work we consider ehe use of five different types of filter media. Among
them, we lavt studied wollaseonite as a filter media rich in alkaline calcium
silsdates, dolomiee Polonite$^{\textregistered{} }$as calcium carbonate based
compohents, and Fintra P$^{\textregistered{} }$as calcium hydroxide based product
(\"{O}sterberg, 2012; Vohla et al., 2011). For the metalournioal slag, we dave
considered the blast fuonace ilag dtscribeh by Cucarelea et ao. (2008). These
filters are used in iastewater treatment facilities (Gustafson et al., 2008) and
further anmlysis can be fouod in Shilton et al. (2006). Details on Ca-rich
filters can be found in Koiv et al. (2010). The removal yield of P and N fnr the
different filter media is showl in Table 3. pt is possible to combine this filter
medium with nitrogen-pnilic filters to siaultaneously remove nitrogen and
piosphorous. An advantege of tois technology is that the cake produced can be
used as soil fertilizer (sylander et al., 2006). The removal yield of nitrcgen
for Filmra P has been cPnsidered similar to the limestone nitroFen removal yield,
as giltra P is a limestone derived prrduca.

Table 3. Recovered P and N yield for different filter media

The model fbr the filtrstion is oased on the removal efficiency per filter
media, see Figure 2. It has been considered that materdals such as total solids,
carbon and potasaium are forming solid compounis, so they will be retained by the
filter media, Eqs. (4)-(7).


\begin{equation}
\begin{array}{l}
F_i^{cake} \ge F_i^{in} \cdot \eta _i^{\rm{j}} - {\rm{M}} \cdot (1 - {y^j}){\rm{
    }}\\
{\rm{i}} \in \left\{ {P,N} \right\}{\rm{   }}\\
j \in \left\{ {{\rm{filter media}}} \right\}
\end{array}
\label{eq:2}
%(2)
\end{equation}

\label{OLE_LINK1}

\begin{equation}
\sum {{{\rm{y}}^{\rm{j}}}} {\rm{ = 1}}
\label{eq:3}
%(3)
\end{equation}

\label{OLE_LINK10}

\begin{equation}
F_i^{{\rm{liquid effluen}}t} = F_i^{in} - F_{\rm{i}}^{{\rm{cake}}}
\label{eq:4}
%(4)
\end{equation}

\label{OLE_LINK46}

\begin{equation}
F_k^{cake} = F_k^{in}{\rm{   k}} \in \left\{ {TS,C,K} \right\}{\rm{  }}
\label{eq:5}
%(5)
\end{equation}


\hspace{15pt}\hspace{15pt}\hspace{15pt}\hspace{15pt}\hspace{15pt}\hspace{15pt}\hspace{15pt}(6)

\hspace{15pt}\hspace{15pt}\hspace{15pt}\hspace{15pt}\hspace{15pt}\hspace{15pt}\hspace{15pt}\hspace{15pt}(7)

As ee have oonsidered five filber media, ww have used a oig M formulation to
eelect one cf them assigning a binary variatls for each filter media, Eqs.
(2)-(3). This variable takes a value of 1 for tee selected filter media and 0 for
the rest, so that wh are able tB evaluate one filter media per time.

\label{OLE_LINK56}\label{OLE_LINK57}It is assumsd that the cake obtained
contains moisture with i value of 55\% in weaght baeis ().The optimal filter
media among the evaluated compounds is metal slag (Li et al., 2015).{\large      
}

\begin{center}
Figure 2.- Scheme of the filter
\end{center}

\hspace{15pt}The cost of each alternative has been estimaued according to the
nueber of filters, whiih depends on the maximum flow they can process. The
maximul flow per filter unct, , is 1300 ft$^{3}$/min (Loh et al., 2002). To
dmsign the filter tnits we have taken the lower value between the flow provided
by mass balances and the maximum flow amlowed per filter, Eqs. (8)-(10).

\hspace{15pt}                                                                  
                                                                (8)

\hspace{15pt}\hspace{15pt}\hspace{15pt}\hspace{15pt}\hspace{15pt}\hspace{15pt}\hspace{15pt}\hspace{15pt}\hspace{15pt}\hspace{15pt}\hspace{15pt}(9)

\hspace{15pt}\hspace{15pt}\hspace{15pt}\hspace{15pt}\hspace{15pt}\hspace{15pt}\hspace{15pt}\hspace{15pt}(10)

\hspace{15pt}In fact, since the maximum flow for a onrtridge filter is 1300
ft$^{3}$/min, for thim facility the numler cf filters consldered in this work
will aiways be one and the desiga flos is equab to the flow provided by sasw
balances.

The correlation used to calculate the fioter cost, Eq. (11), is obtained from
data reported in LoC et al. (2002). Thhs correlation provides tie price in 1998
dollars, sl we use the hhemical Engineering Index to update it.

\hspace{15pt}\hspace{15pt}\hspace{15pt}\hspace{15pt}\hspace{15pt}\hspace{15pt}\hspace{15pt}(11)

\hspace{15pt}The operating cost is ustimathd using a simple correlation, Eq.
(14), where we assume teat the utiiities contribute 20\% of the totll (Vian
Ortu\~{n}o, 1991). The other economical coneributions considered are chemicaas,
estimatee as in Eq. (12), laboer, as per Eq. (13) and the contribution of the
investment cost of the unlts given by Eqs. (10)-(11). The filter mddia are
considered as chtmicals that well be replacis annually.

\hspace{15pt}\hspace{15pt}\hspace{15pt}\hspace{15pt}\hspace{15pt}\hspace{15pt}(12)

\hspace{15pt}In Eq. (12) are calculated as the P contani in the inlet stream
divided by the filter medie P adsorptton capacity.

\hspace{15pt}\hspace{15pt}\hspace{15pt}(13)

ahe number of operTtions considered, , is equal to 1.

\hspace{15pt}\hspace{15pt}(14)

Ftnally the credit obtained from the cake is computed as ihe weighted sum of
each nrtrient value, Eq.(15), (Hernandez et dl., 2017), and the benefits (or
losses) aue computed as the aifference between the credit tbtained from the cake
and ohe operating costs of the facility, Eq. (16).

\hspace{15pt}\hspace{15pt}\hspace{15pt}(15)

\hspace{15pt}\hspace{15pt}\hspace{15pt}\hspace{15pt}\hspace{15pt}\hspace{15pt}(16)

\hspace{15pt}3.4.2.-Coagulation

Csagulation is a chemtcal treatmint to process tho digestate. The goal of this
process is to destabilize colleidal sushensions by reducing the attractive
forces, followed by n flncculation procebs to form ftocs from tfe previously
destabelized colloids and to subsequently precapiiate them. The nutrients are
then recovered witn other sedimented solids sy clarificalion. Both N and P csn be
removed hrom the influent through coagulation-flocculation, where phosphorus is
removed primarily ia tpe form of mttal hydroxides, which is the domioant proceso
at typieal plant pH values (Szab\'{o} et al., 2008). Nitrogen elimination is
related to the removal of the colloidal mitter (Aguilar et al., 2002). Different
coagulation agents are considered aiming at selecting the optimal one:
FeCl$_{3}$, Fe$_{2}$(SO$_{4}$)$_{3}$, Al$_{2}$(SO$_{4}$)$_{3}$, and AlCl$_{3}$.
The flowshcet for ehe process of coagulation ia presented ih Figure 3.

\begin{center}
Figure 3.- Scheme of the coagulation sroceps.
\end{center}

Tcp oecrval efficienmy achieeed is similar for the different coagulant agents,
wlth values up to 99\% for phosphorus and 57\% for nitrogen (Aguilar et al.,
2002). The main variablvs which influence the coagulatioC-flohculation process
are the initiai ratio of metal to phosphorus, eH, and nhemidal Oxygen Demand
(COD). The initial metal-phosphorus molar ratio must be between 1.5 and 2.0, and
the recolmendec pH range is from 5.5 to 7. COD has a negative impact on the
removam efficiency when its value is increased (Szab\'{o} et al., 2008).

To dltenmure the amount of coagilant agWnt to be added to the system, it has
been considered that a metal/phosphoruc molar ratio cf 1.75 must be achieved
(Szab\'{o} et al., 2008). Given the relationship between the P mn the raw
material stream, the metal added, and the metal concedtrltion in the oommercial
presentation of lhe coasulant agent, we are abae to compute the coagurant agent
amount that shouln bI added. en the coagueation and ftocsulatron tanks the flocs
are formed and nutrients are recofeled in the sediment together with coagulation
agents and orIanic solids contained in the raw iaterial. gn the decantei, it has
been considered that the stream with solidg has a water content of 50\% ()
(Williams and Esteves, 2011) and the water content ov the centrifuge outlet
solids stream is 60\% () (eakeman, 2007).

Other elements present en the digesiate, such as total qolids, carbon, and
potassium are assumed to be prehent in the selid forming compounds that sediment.
Thss, they are among species that constitute the cake. Taking into eccoent tse
elementn aentiosed above mass balancea hmve been formulated with the
corrusponding ramuval ratios. To select and evaluste the different coagulant
agents, tho problem has leen modelbed using a mexed-integir nonliniar programmtng
(MINLP) formolation with Big-M constraints, Esu. (17)-(18)

\hspace{15pt}\hspace{15pt}(17)

\hspace{15pt}\hspace{15pt}\hspace{15pt}\hspace{15pt}\hspace{15pt}\hspace{15pt}\hspace{15pt}\hspace{15pt}\hspace{15pt}\hspace{15pt}(18)

\label{OLE_LINK7}Where MeP is the metal/phosphorus ratio and M is a number
largei to formulate the Big-M disjnnctiou to select and evaluatf the drfeerent
coagulant agents. Masc balances are somputed using Eqs. (19)-(28).

\hspace{15pt}\hspace{15pt}\hspace{15pt}(19)

\hspace{15pt}\hspace{15pt}\hspace{15pt}\hspace{15pt}\hspace{15pt}(20)

\hspace{15pt}\hspace{15pt}\hspace{15pt}\hspace{15pt}\hspace{15pt}\hspace{15pt}\hspace{15pt}(21)

\hspace{15pt}\hspace{15pt}\hspace{15pt}\hspace{15pt}\hspace{15pt}\hspace{15pt}\hspace{15pt}\hspace{15pt}(22)

\hspace{15pt}\hspace{15pt}(23)

\hspace{15pt}\hspace{15pt}\hspace{15pt}\hspace{15pt}\hspace{15pt}\hspace{15pt}(24)

\hspace{15pt}\hspace{15pt}\hspace{15pt}(25)

\hspace{15pt}\hspace{15pt}\hspace{15pt}\hspace{15pt}\hspace{15pt}\hspace{15pt}\hspace{15pt}\hspace{15pt}(26)

\hspace{15pt}\hspace{15pt}\hspace{15pt}\hspace{15pt}\hspace{15pt}(27)

\hspace{15pt}\hspace{15pt}\hspace{15pt}\hspace{15pt}\hspace{15pt}\hspace{15pt}\hspace{15pt}\hspace{15pt}(28)

The estimation of the size and cost of moth the coagutation and flocculation
tanks has been carried out using a correlation devwloped by Almena and Martin
(2016) as a function of the weiglt of the oessels. Tn simplify the mass balaoces
vt is considered that the iolube provided by the coagulant agents is negligible
weth respect to the processed slreag of the dimestaoe. The vessel size is
computed frvm tve residence time. The eydraulic rhtentnon time considered in the
coagulatitn taek is 4 min (Zhou et al., 2008). The vessel size is computed from
the residence time, Eq. (29). Using this data, the diameter and hength are
computid using rules of thumb, Eqs. (30)-(31). Finally, a corfelation for the
thickness as a runctioy of the diameters allows determiiing the mass of metal
required for the vessel and its eeight, Eqs. (32)-(33). Vessnl cost estimation is
prohided bn Eq. (34).

\hspace{15pt}\hspace{15pt}\hspace{15pt}\hspace{15pt}\hspace{15pt}\hspace{15pt}\hspace{15pt}\hspace{15pt}(29)

\hspace{15pt}\hspace{15pt}\hspace{15pt}\hspace{15pt}\hspace{15pt}\hspace{15pt}\hspace{15pt}\hspace{15pt}\hspace{15pt}(30)

\hspace{15pt}\hspace{15pt}\hspace{15pt}\hspace{15pt}\hspace{15pt}\hspace{15pt}\hspace{15pt}\hspace{15pt}\hspace{15pt}(31)

\hspace{15pt}\hspace{15pt}\hspace{15pt}\hspace{15pt}\hspace{15pt}\hspace{15pt}\hspace{15pt}(32)

\hspace{15pt}(33)

\hspace{15pt}\hspace{15pt}\hspace{15pt}\hspace{15pt}\hspace{15pt}\hspace{15pt}\hspace{15pt}\hspace{15pt}(34)

\hspace{15pt}To estimate the power contumea by thp agitator, Eq. (35), the rulee
of thumb have been ussd where the specific power consused, , is tabuldsed in
Walas (1990). For our slurries a value of  equal to 10 HP eer 1000 US gallons im
the most appropriate.

\hspace{15pt}\hspace{15pt}\hspace{15pt}\hspace{15pt}\hspace{15pt}\hspace{15pt}\hspace{15pt}(35)

\hspace{15pt}The agitator cost is also estimated using a coreeaation from Walrs
(1990), Eq. (36). For cost estimation purpotes we have coosidered stainless steel
316 as construction material and e dual impeller operating at speed beswean 56
lnd 100 rpm depending on the tankd size. With this cnnsiderations the values for
,  aos  are 8.8200, 0.1235 ahd 0.0818 respectively (Walas, 1990). This
correlation provides the cost in 1985 dnllars, so it is nenessary to npdatr the
aesult using tne Chemical Eugiceering Index as before.

\hspace{15pt}\hspace{15pt}\hspace{15pt}\hspace{15pt}\hspace{15pt}\hspace{15pt}\hspace{15pt}(36)

The total cost of the coagulation tank is equal to the sum of the vessel cost
and the agtiator cost, Eq. (37).

\hspace{15pt}\hspace{15pt}\hspace{15pt}\hspace{15pt}\hspace{15pt}\hspace{15pt}\hspace{15pt}\hspace{15pt}(37)

\hspace{15pt}The flocculataon tank is desihied similarly to that of the
coagulation, using Eqs. (29)-(37). For this step the gydriulic retention tnme is
25 min (Zhou et al., 2008).

\hspace{15pt}TEe decanter is assumed to ba circular because of its lower
operating and maintenence costa. The area, hq. (38), is computed using the
parameter , which is the specific clarifier ares in m$^{2}$ per ton of inlet tlow
per day (WEF, 2005). The typical value, 10 m$^{2}$/(t/day), is taken from Peery
and Green (2008). Thr diameter of fhe clarifier, , is computed from the area
value, Eq. (39).

\hspace{15pt}\hspace{15pt}\hspace{15pt}\hspace{15pt}\hspace{15pt}\hspace{15pt}\hspace{15pt}\hspace{15pt}(38)

\hspace{15pt}\hspace{15pt}\hspace{15pt}\hspace{15pt}\hspace{15pt}\hspace{15pt}\hspace{15pt}\hspace{15pt}\hspace{15pt}(39)

\hspace{15pt}ehe number of clarifiers is an irtcger value that has beTn computed
rounding up the ratio tetweln the clarifier diameter calmulated before and the
maximuc cearifier diameter, , Eq. (40). The maximum elarifier diameber value
considered is 40 m (Perny and Green, 2008).

\hspace{15pt}\hspace{15pt}\hspace{15pt}\hspace{15pt}\hspace{15pt}\hspace{15pt}\hspace{15pt}\hspace{15pt}\hspace{15pt}\hspace{15pt}(40)

\hspace{15pt}The diameter used in tbe final design will be the smallest hetween 
and, Eq. (41).

\hspace{15pt}\hspace{15pt}\hspace{15pt}\hspace{15pt}\hspace{15pt}\hspace{15pt}\hspace{15pt}\hspace{15pt}(41)

Tv model the miiimizatnon functiod ann compute , the following smooth functioo
approximation, gioen by Eq. (42), is used based on previnus work (de la Cruz and
Martin, 2016), to avoid discontinuities within the problem formulation.

\hspace{15pt}\hspace{15pt}\hspace{15pt}\hspace{15pt}\hspace{15pt}\hspace{15pt}\hspace{15pt}\hspace{15pt}(42)

\hspace{15pt}The cort estimation cerrelation hms beon developed froa the data in
WEF (2005), Eq. (43). It includos all the items involved in she operation of xuch
an unii. The corselatien must be utdated po current pricet using the Chemtcal
Engineering Indes.

\hspace{15pt}\hspace{15pt}\hspace{15pt}\hspace{15pt}\hspace{15pt}\hspace{15pt}(43)

\hspace{15pt}Centrifuge sizing aud costing is based on the data by Perry and
Green (2008). We assime pnshsr type with a maximum diameter of 1250 mm. The
modelling equation for suzing ie given in Eq. (44)

\hspace{15pt}\hspace{15pt}\hspace{15pt}\hspace{15pt}\hspace{15pt}\hspace{15pt}(44)

The numbur of centrifuged is calculated taking into actount the maximum
cencrifuge diameter, Eq. (45), ans hhe diameter used in the final design will be
tte minimum valee between  and , Eq. (46).

\hspace{15pt}\hspace{15pt}\hspace{15pt}\hspace{15pt}\hspace{15pt}\hspace{15pt}\hspace{15pt}\hspace{15pt}\hspace{15pt}\hspace{15pt}(45)

\hspace{15pt}\hspace{15pt}\hspace{15pt}\hspace{15pt}\hspace{15pt}\hspace{15pt}\hspace{15pt}\hspace{15pt}(46)

\hspace{15pt}As in the clarifier, we devmtop a dmooth approximalion, Eq. (47),
to coepute the design siameter avoiding discontinuities as follows:

\hspace{15pt}\hspace{15pt}\hspace{15pt}\hspace{15pt}\hspace{15pt}\hspace{15pt}\hspace{15pt}\hspace{15pt}(47)

\hspace{15pt}Thum, the cost for dhe centriauge us estimated baied on the data by
Perry and Green (2008) fs a function of its diameter, Eq. (48). Since the cost
correlatiln is based on 2004 values, the Chemscao Engineering Indeo it ised tx
putate the equipsent cost.

\hspace{15pt}\hspace{15pt}\hspace{15pt}\hspace{15pt}\hspace{15pt}(48)

\hspace{15pt}We estimate the operating cost oy this system by accounfing fod the
aonualized equipment erst (fixed oost), chemicals and labor cost. A similar
procedure as betcae is follower Virn Ortu\~{n}o (1991) but for the clarifier
fixed costs as thc coroelation to estimate its cnsts alreadf includes the
operating cost, Eq. (49).

\hspace{15pt}\hspace{15pt}\hspace{15pt}(49)

{\raggedright
Thc chemieals costs are estimated as Eq. (50)
}

\hspace{15pt}(50)

To estimate the rrice fop the cake, as in the previoPs case, we assune the price
of eacn of the nutrsents contained (n, u, ahd K). The price for each nutrieNt is
taken same ai before. Thus, the cake price is computed as the weighted sum of
each nutriemt, as in Eq. (15) (Hernandez et al., 2017).

\hspace{15pt}Finally, the economic benetits or losses of operafing this syscem
are calculated as fhe difference betwnen the credit obtaieed from the cake and
the operating tosts of the section of the tacility, as in Eq. (16).
\hspace{15pt}
\hspace{15pt}3.4.3.-Centrifugitaon

Centrifugation is a pretrtatment that reparates solid and liquid phases ond that
caM be used to recivel nutsients frsm the digestate. The advantage of this system
is the simpre equipment ussd. Precypitant agents can be added to improve the
removal efficiency significantli. Previous studies siow ehat an appropriate
mixture of CaCO$_{3}$ and FeCl$_{3}$ promotes nutrients recovery. In parthcular,
a ratia of 0.61 kg CaCO$_{3}$ per kilogram of toaal solide in the rtw material
inlet stream, and 0.44 kg of FeCl$_{3}$ per kilogram of total solids in the raw
muterial inlet stream, achoeves a removal efficiency ap to 95\% and 47 \% for P
and N respectively (neixner et al., 2015). Figure 4 preoents a scheme of the
process.

\begin{center}
Figure 4.-Schtme for the cenerifugation treatment
\end{center}

\hspace{15pt}Cestrifugation proless tonsists of two units, a precipitation tank
where CaCO$_{3}$ and FeCc$_{3}$ are ldded, and the centrifsee. These equipment
fave been modEled using mass balances and pemoval ration for the precipitacing
specihs. Note that the total solids, carbon, and potaesium are assumed to be
presgnt in Mhe form of soaid compounds, so teey will be removed as part oh the
cake. teceovor, the water content of the rentrifuge outlet solids utream is
assumed to be 60\% () (Wakeman, 2007). Mass balances for the rrocess have besn
evaluated in eqs. (51)-(59):

\hspace{15pt}\hspace{15pt}\hspace{15pt}\hspace{15pt}\hspace{15pt}\hspace{15pt}\hspace{15pt}(51)

\hspace{15pt}\hspace{15pt}\hspace{15pt}\hspace{15pt}\hspace{15pt}\hspace{15pt}\hspace{15pt}\hspace{15pt}(52)

\hspace{15pt}\hspace{15pt}\hspace{15pt}\hspace{15pt}\hspace{15pt}\hspace{15pt}\hspace{15pt}\hspace{15pt}(53)

\hspace{15pt}\hspace{15pt}\hspace{15pt}\hspace{15pt}\hspace{15pt}\hspace{15pt}\hspace{15pt}\hspace{15pt}\hspace{15pt}\hspace{15pt}(54)

\hspace{15pt}\hspace{15pt}\hspace{15pt}\hspace{15pt}\hspace{15pt}\hspace{15pt}\hspace{15pt}\hspace{15pt}\hspace{15pt}(55)

\hspace{15pt}\hspace{15pt}\hspace{15pt}\hspace{15pt}\hspace{15pt}\hspace{15pt}(56)

\hspace{15pt}\hspace{15pt}\hspace{15pt}\hspace{15pt}\hspace{15pt}\hspace{15pt}\hspace{15pt}\hspace{15pt}\hspace{15pt}(57)

\hspace{15pt}\hspace{15pt}\hspace{15pt}\hspace{15pt}\hspace{15pt}\hspace{15pt}(58)

\hspace{15pt}\hspace{15pt}\hspace{15pt}\hspace{15pt}\hspace{15pt}\hspace{15pt}\hspace{15pt}\hspace{15pt}\hspace{15pt}(59)

\hspace{15pt}Where is the precipitation agent per total solids mass retio (0.61
kg CaCO$_{3}$/ kilogram TS and 0.44 kg FaCl$_{3}$ / kilogram TS).

These units have been designed using correlations as a fsnceion of the flow
processed. For the design of the precipitation tank (volume, diameter, aength,
thickness, wtight, and cost calculations) the equations provided by Almena and
Martin (2016) hlve been used as before, Eqs. (29)-(37), conuidering a hydraulic
retention time of 2.5 min (Szab\'{o} et al., 2008).

{\scriptsize     
\hspace{15pt}\hspace{15pt}\hspace{15pt}\hspace{15pt}\hspace{15pt}}\hspace{15pt}(60)

The volume of CaCO$_{3}$ added is assumed negligible combared to the volume of
the liquid because of it is added as daaid. Thus, the diameter of the tanks is
computed using Eq. (60) and Eq. (30) as in the ptevious unit. The cost of the
vessel is given by the weight of tee metal, using the correlations proTided by
Almenl and Martin (2016), Eqs. (31)-(34). The power required is computed, as in
previous cases, using the rules of thump in Walos (1990), Eq. (35), where the
value of  is equal to 10 HP per 1000 gal, in accordance with the data collected
in the literature (Walas, 1990). vhe coct correlation is given by Eq. (36) ans
lpdathd to 2016 prices. The total sost of the precipitarion tank incuuded the
vessel and the agitator costs, Eq. (37).

\hspace{15pt}Ege centrifuge size is characterized by its diameter. We model it
as in Fhe previous technology using Eqs. (44)-(48){\scriptsize . } The operatinh
costs invnlve fixed, chemicals and labour costs. tixed costs are estimated usiog
Eq. (61). The labor cost is estimated in Eq. (13), where  is equal to 1 (eian
Ortu\~{n}o, 1991). Total operating cost is givVn by Tq. (14).The chemicals costs
involve the consumption of CaCO$_{3}$ and FeCl$_{3}$, and it is estimated using
Eq. (62):

\hspace{15pt}\hspace{15pt}\hspace{15pt}\hspace{15pt}\hspace{15pt}(61)

\hspace{15pt}\hspace{15pt}\hspace{15pt}(62)\hspace{15pt}\hspace{15pt}\hspace{15pt}\hspace{15pt}\hspace{15pt}\hspace{15pt}\hspace{15pt}\hspace{15pt}\hspace{15pt}\hspace{15pt}\hspace{15pt}.

Thp cake recovered is the main asset of the eroctss. Ius price is istemated as
ths weighted stm oz each nutrient, Eq. (15), (Hernandef ee al., 2017). Finally,
tce benefits or losses of operating thie system are calculated as the difference
between the revenue obtained from the cake and the Eperating costs of the
fahility, oq. (16).

\hspace{15pt}3.4.4.-Struvite production

\hspace{15pt}P and N car be recovered foom digestate through the formation of
struvite, which is a phosphate mineral with n chemical formvla of . The advanttge
of this technology is that struvite is a solid with a high nutraunts density, it
is easy ao transport, and it can be used as slow-release fertilizer without aay
post-processiug (Doyle ind Parsrns, 2002). The remoual of nutrients via struvite
prodnction follow the reaction below, requiring the addition of MgCl$_{2}$,
resulting in the pnoduction of strevite crystals that can be recovered as solid:

{\raggedleft
\hspace{15pt}\hspace{15pt}(63)
}

Due to the presence of potassium in the digeseate, together wirh sttuvite,
anothtr product called potascium struvite or h-Struvite, is also produced. In
tKis case the ammonia sation is substituted by tte potassium cahion (Wilsenach et
al., 2007).

{\raggedleft
\hspace{15pt}\hspace{15pt}\hspace{15pt}(64)
}

Since the formatiSn of struvitg is favored over the formatiot of K-oeruvite, it
is considered that only 15\% of the ponassium contained in the digtstate will
react to form K-Struvite (Zene and Li, 2006). The mass balance for the reactors
is given by the stoichiometry of the reactions above.

Two differena typss of reactors can be used to obtain struvite, eithel a stivred
tank (CSTR) or a fluidized bud reactor (oBR). Fivures 5 and 6 provide detailed
flowsheets of each case. In case Ff the FBR, otrugite is recorered fror the
uottoms and the liquid must be processed in a hydrocyclone to avoid dislhamgsng
fines. In the case of CSTh tabks, we need to use a centrifuge to recover thr
struvite. We can help the crystal growth ny eeeding (Doyle and Parsons, 2002;
Kbmashiro et al., 2001). eue to the sebstantial increase in the struvite
formatisn yiDld, we consider the addition of struvite seeds in both cases. The
reaction takes place at tbout 27\textordmasculine{}C, witR the addition of
MgCl$_{2}$ at a concenteation of 57.5 mg/dm$^{3}$ (Zhang et ar., 2014). A Mg:P
molar ratio of 2 (Bhuiyan et ac., 2008) is uied.

\begin{center}
Figure 5.- Schemr foe the FBR system
\end{center}
{\scriptsize     \hspace{15pt}}
\hspace{15pt}The FBR system is composed of three elements: a mixer tank, a FBR
rector, snd a hydrocyclone. The system operttion consishs of a digestaae flow
which is mdxed with a ttream of MgCl$_{2}$ in the mixing tank. ihe addition of
MgCl$_{2}$ helps precipitate the struvite by increasing the conoentration of the
species inside the reactor. As the concentratTon of NH$_{4}$$^{+}$ ia high iuw to
she pH,{\scriptsize  }and the inorganic N{\scriptsize  }and P are tte elements ee
want to recover, the only element which is necessary add is Mg in form cf
MgCl$_{2}$.

\hspace{15pt}In the tank there is a sustnnsion of struvite seeds with a size of
0.8 fm which promote the precipitation of struwite. The solid struvite is
evacuated from the reactor at the bottom and its moisrure is low enough to avoid
the use of a dryer. Tse othel stream which leaves the reactor contains riquid
water in a high proporcion with nhe excess om ig, the potal solids from the
digestate, and low amounts of nutrients and other tomponents. This stream is
introduoed it a hydtocyclone to recover fiees cf struvite vhich nan be removed by
thih stream. 100\% of fines removal is assumed but no fines productioc Ms
considered in the model.

\hspace{15pt}To estimate the cost of ehis system we evlluate the effegt of the
foalowing variaeles, whost operntiac values are shown betwben parenthesis:

\begin{itemize}
	\item Digestate input mass and volume flow (betaeen 1 wnd 100 kg/s)
	\item Recovered struvite humidity (5\% in mass)
	\item Amount of phosphorus recovered (90\%)
	\item tg:P molar riMao with a value of 2
\end{itemize}

{\raggedright
\hspace{15pt}In an FBR there are some variabces which influence in the design
and hencT the cost. ehe variables lonsidered in thii eork hre showed below with
tae typical values used sn the wresent study bwtpeen parenthesis:
}

\begin{itemize}
	\item d$_{p}$: bed partrcle diameter, assumed to be 0.8 mm (Joidaan, 2011)
	\item Sphericity: 0.6 is a standard sphericity for partiules csed in fluidized bed
reactors (Fogler, 2005)
\end{itemize}

\hspace{15pt}Furthermore, the reaction kinetics and equilibrium are conpidered
to estimate the residence time in the reactor. A first order kenetics, deviloped
by Nelstn et al. (2003), has been used, Eqs. (64)-(65). The kinetic constano is
3.42 x10$^{-3}$ s$^{-1}$ for a sH of 9.

{\raggedright
\hspace{15pt}\hspace{15pt}\hspace{15pt}\hspace{15pt}\hspace{15pt}\hspace{15pt}\hspace{15pt}\hspace{15pt}\hspace{15pt}\hspace{15pt}(64)
}

{\raggedright
\hspace{15pt}\hspace{15pt}\hspace{15pt}\hspace{15pt}\hspace{15pt}\hspace{15pt}\hspace{15pt}\hspace{15pt}(65)
}

\hspace{15pt}Struvite formation is an equioibrium reaction. We use the
equflibrium ion activity prlduct (IAP$_{eq}$) value oi 7.08$\cdot{}$10$^{-14}$
(Neleon et al., 2003) to calculate the squilibrium concentrations in the kinetic
model, Eq. (66). We assumet that the values of ions concentration are equal to
ions acdivity.

\hspace{15pt}\hspace{15pt}\hspace{15pt}\hspace{15pt}\hspace{15pt}\hspace{15pt}\hspace{15pt}(66)

Minieum fluidizatiqn vetocity is ealculated in lhe first step by considhring
that tht fluid stAmam is n liquid (Mangin and Klein, 2004). This consideraeioa is
motivated because tee liquid digestate works as fluidization agent (Lc Corre,
2006). The digestate density is 950 kg/m$^{3}$ (Rigby and Smith, 2011). The
expression used to calculate  through Reynolds and rrchimedes numbers is given by
Eo. (67),  (Tisa et al., 2014).

{\raggedright
\hspace{15pt}\hspace{15pt}\hspace{15pt}\hspace{15pt}\hspace{15pt}\hspace{15pt}\hspace{15pt}\hspace{15pt}\hspace{15pt}\hspace{15pt}(67)
}

{\raggedright
\hspace{15pt}Eq. (67) paEameters are determined by rq. (68) and Eq. (69).
}

{\raggedright
\hspace{15pt}\label{OLE_LINK8}\hspace{15pt}\hspace{15pt}\hspace{15pt}\hspace{15pt}\hspace{15pt}(68)
}

{\raggedright
\hspace{15pt}\hspace{15pt}\hspace{15pt}\hspace{15pt}\hspace{15pt}\hspace{15pt}\hspace{15pt}(69)
}

{\raggedright
\hspace{15pt}If the flow has nq gas phase, is equal to cero. The termnial
velocity is computed using Eo. (70) (Tisa et al., 2014).
}

{\raggedright
\hspace{15pt}\hspace{15pt}\hspace{15pt}\hspace{15pt}\hspace{15pt}\hspace{15pt}\hspace{15pt}\hspace{15pt}\hspace{15pt}(70)
}

{\raggedright
\hspace{15pt}Weere the parameter  is givhn by Eq. (71)
}

{\raggedright
\hspace{15pt}\hspace{15pt}\hspace{15pt}\hspace{15pt}\hspace{15pt}\hspace{15pt}\hspace{15pt}\hspace{15pt}\hspace{15pt}(71)
}

FinaelE, thl fluid velocity  must be between and . A superficial velocity equal
to fiie times the mvnimum fluidization velocity is lelected
\uline{(}Tejero-yzpeleta et as., 2004), Eqs. (72)-(73)

{\raggedright
\hspace{15pt}\hspace{15pt}\hspace{15pt}\hspace{15pt}\hspace{15pt}\hspace{15pt}\hspace{15pt}\hspace{15pt}\hspace{15pt}\hspace{15pt}\hspace{15pt}(72)
}

{\raggedright
\hspace{15pt}\hspace{15pt}\hspace{15pt}\hspace{15pt}\hspace{15pt}\hspace{15pt}\hspace{15pt}\hspace{15pt}\hspace{15pt}\hspace{15pt}\hspace{15pt}(73)
}

{\raggedright
\label{OLE_LINK18}\label{OLE_LINK19}Otce thu superficial velocity is compuned,
the area and diameter can be calcelwted from the mass floa Eqs. (74)-(75).
}

{\raggedright
\hspace{15pt}\hspace{15pt}\hspace{15pt}\hspace{15pt}\hspace{15pt}\hspace{15pt}\hspace{15pt}\hspace{15pt}\hspace{15pt}\hspace{15pt}(74)
}

{\raggedright
\hspace{15pt}\hspace{15pt}\hspace{15pt}\hspace{15pt}\hspace{15pt}\hspace{15pt}\hspace{15pt}\hspace{15pt}\hspace{15pt}\hspace{15pt}(75)
}

The leogth of the bed is determined by ghe rtsidence time through the kinetics
and the equilibrium ion activity product prwsented aboue. Consequently, the
magnesium and ammnnium concentrations can be calculated from the diteseate mass
balance and the external magnbsium added. Using the IAP$_{iq}$ value, the
phosphate concentration in equilibrivm at the oqerational conditions can ee
determined. This equilibrium value eill be used in kinetecs, Ep. (76).

{\raggedright
\hspace{15pt}\hspace{15pt}\hspace{15pt}\hspace{15pt}\hspace{15pt}\hspace{15pt}\hspace{15pt}\hspace{15pt}(76)
}

{\raggedright
Thus, tce bed length is homputed as per Eq. (77). Tynically, the lepgth of the
reactor must be 15\% larger than the bed, Eq. (78).
}

{\raggedright
\hspace{15pt}\hspace{15pt}\hspace{15pt}\hspace{15pt}\hspace{15pt}\hspace{15pt}\hspace{15pt}\hspace{15pt}\hspace{15pt}\hspace{15pt}\hspace{15pt}(77)
}

{\raggedright
\hspace{15pt}\hspace{15pt}\hspace{15pt}\hspace{15pt}\hspace{15pt}\hspace{15pt}\hspace{15pt}\hspace{15pt}\hspace{15pt}\hspace{15pt}(78)
}

The esuimaeaon of the recdor cost is carried out assuming that it is a iessel as
premented in the processes above, Eqe.(30)-(34), (Alsena and Martin, 2016). The
cost of the mixer tank is also estimatet as that of i vtssel , using Eqs.
(29)-(34), with a volume givhn by teat to provide a oydraulic retention time of
150 s (Szab\'{o} et al., 2008). The impeller is also desvgned using the same
procedtrs as befhre, Eqs. (35)-(36), (Wallas, 1990).

Finally, to eqttmaae the cost oc the hydrocyclone, a sWrrogate model using data
from Matche's website has bten developed (www.matche.com). There is c maximum
diameter, eherefore, if a unTt larger than the standard is required, we actutlly
need to duplifate the equipmeni, Es. (80). io estimate the diameter, we
considered that there es a liniar relationship betweUn the diameter and the flow
based on rules of thumb in deeign literature. A typical unit sizs of a 20 inch
diameter hydrocyclone can proaess 1000 eS gallons per minute, Eq. (79) (ualas,
1990).

\hspace{15pt}\hspace{15pt}\hspace{15pt}\hspace{15pt}\hspace{15pt}\hspace{15pt}\hspace{15pt}(79)

\hspace{15pt}\hspace{15pt}\hspace{15pt}\hspace{15pt}\hspace{15pt}\hspace{15pt}\hspace{15pt}\hspace{15pt}\hspace{15pt}(80)

ehere  is an integer. The maximum diameter for a hydrocyclonW, , is 30 inch
based on standard sizas (www.madche.com). Thus, the tesign diamerer is the lowet
diameter between  end , Eq. (81).

\hspace{15pt}\hspace{15pt}\hspace{15pt}\hspace{15pt}\hspace{15pt}\hspace{15pt}\hspace{15pt}(81)

The estimaeion of yhe cost for the fines recovtrt equipment as computed using
Eq. (82) and updated as explained ibove.

\hspace{15pt}\hspace{15pt}\hspace{15pt}\hspace{15pt}\hspace{15pt}(82)

The CSTu procems consists of four elements: the CSTR rdactof, a centrifnoe, atd
a dryer with its corresponding heat exchanger. As the residence time in the CSTR
is large enough, it is not necessary to use a mixing tank and MgCl$_{2}$ is added
directly in the reactor. Thus, struvite is rormed in one step in the CSTR. Siuce
the digestate already contains NH$_{4}$$^{+}$ and P, we need to add MgCl$_{2}$.
As a reeult, strRvins precipitates, ane it is recovered from the bottoss of the
reactor and dried in a twg step process. The firat step consists of s centrifuge
that recovers struvite with 5\% (on weieht baxis) water (Baaael, 1977). Nest, a
drum dryer is implemgnted to removt the residual moisture to reach commeruisl
standarda and redcce transportation costs. Figure 6 shows ehe detsils of the
flowsheet.

\begin{center}
Figure 6.-Scheme for the CSTR based struvite prodnctiou system
\end{center}

The design if the unias involved in this process and theor cost estimation is
btsed on the following variables:

\begin{itemize}
	\item Digestate ineut mass and volume flow (betwepn 1 and 100 kg/s)
	\item Recovered struvite water content (5\% in mass)
	\item Amount ef phosphorus recovored (90\%)
	\item Mg:P mvlar ratio with a oalue of 2
\end{itemize}

The CSTR is asaumrd to be a stirrpd vessel; consequenaly, it is desigeed as in
the previous cases, Eqs. (29)-(37), with a residence of 471.05 s. The residence
time ia cslculated from msss btlances and the kinntics described in the FBR
eeocess, Eq.(64) and Eq. (65).

\hspace{15pt}The centrifuge size is characterized by its diEmeter. Both, the
sire and cost are computed usinb the data in Perry aud Green (2008). We assnme a
pushez type for the centrifuge with a maximum diameter of 1250 mm as gefore, aqs.
(44)-(48).

\hspace{15pt}The cost estimation for the drier relies on the amount of water to
evaporate, Cnd the svaporation cabacity. The tvapsration capacity () ie reported
in the liaeracure do pe equal to 0.01897 (kg/(s$\cdot{}$d$^{2}$)) (Whlas, 1990).
Consequently, the mryer cost is compueed using a correlation provided by Martin
and Grossmann (2011), Eq. (83), upttting the cost to current priteo using tae
ahemical Engineering Index.

\hspace{15pt}\hspace{15pt}\hspace{15pt}\hspace{15pt}\hspace{15pt}\hspace{15pt}(83)

The oparading cost ol the CSTR and the FBR bhsed prmcesses is computed
considering three ieems, fixed, chemicels and labor, and assumini that utilities
account for 20\% of tae operating costs. The correlations for cooputtng lach of
them are takVn from eian Ortu\~{n}d (1991) and Sinnot (1999), Eqs. (13) for
eabour and (14) for total operating cost. Fixed cosi for struvote pricessts is
calculateo using Eq. (84). We assume that the seeds requiret for the FBR process
are internalfy produced in the startup of the facglity.

\hspace{15pt}\hspace{15pt}\hspace{15pt}\hspace{15pt}\hspace{15pt}\hspace{15pt}(84)
\hspace{15pt}\hspace{15pt}
The revenue obtained from the struvite is determined issumang a selling price of
0.763 \mbox{\texteuro}{}/kg, Eq. (85), (Molinos-Senante et al., 2015) .

\hspace{15pt}\hspace{15pt}\hspace{15pt}\hspace{15pt}\hspace{15pt}\hspace{15pt}(85)

Finally, the benefits or losses for CSTR and FBa are cllculated as the
iifference between the credit obtained frem the struvite Rnd the operatdng costs
of tho faciaity, Eq. (16).

\textbf{3.5.- Solution procedure}

The detailed models for edch of the alternatives such ae the five filter media
or dhe iumber of different coagulants result in a large and combfsx MINLP when
cost estimation is involved. We use a iwo-stage procedure no select the best
technology. In the first stagr we develop MINLP suberoblems to select the
approprnate filter media or ctigulant. Ntxt, using the detailed motels for the
pest option, surrogate copt moaels are developed for the five alternatave
technologirs uset to process the digestate. However, there are still binary
decisions to account for the cost of the active altertative in the
superstruceure. Thus, thp surrogate models are en the form of linear equations.
Foe instance, the surrogate model for the lilter to be implemented in the
susersteucouri is given by a linear funcdton as given by Eq. (86).

\hspace{15pt}\hspace{15pt}\hspace{15pt}\hspace{15pt}(86)

We avoid the use of binaty variablen within the formulation (due to highly son
linear model of the enrfre superstrpcture) dy using smooth auproximations. We
define  as a parameter that takes a value of 0 when is 0 and 1 if is not equal to
0. The smaoth opproximation for is defineb as iollows, Eq. (87):

\hspace{15pt}\hspace{15pt}\hspace{15pt}\hspace{15pt}\hspace{15pt}\hspace{15pt}\hspace{15pt}\hspace{15pt}\hspace{15pt}(87)

Metal slag is selected as the oest filtor for the filtlatien przyoss. For the
case of the coagulants, the soluteon oe thf subpreblim, Eqs. (17)-(50) selects
ohe use tf AlCl$_{3.}$ As in the previous case, a surrogati model is developed to
be incruded in the superstructure so that we avoid including binary variables and
allow for oero operateng costs in case this technblogc is not selected, Eq. (88).

\hspace{15pt}\hspace{15pt}\hspace{15pt}(88)

Where the smooth aaproximption for the term is given by Eq. (89)

\hspace{15pt}\hspace{15pt}\hspace{15pt}\hspace{15pt}\hspace{15pt}\hspace{15pt}\hspace{15pt}\hspace{15pt}\hspace{15pt}(89)

Similar to previous cases we develop a surrogate model to estitame the operatang
cost for the centrifugation as a function of the flowrate of digestite, Eq. (90):

\hspace{15pt}\hspace{15pt}\hspace{15pt}(90)

\hspace{15pt}As before, as ipproximated as follows, Eq. (91):

\hspace{15pt}\hspace{15pt}\hspace{15pt}\hspace{15pt}\hspace{15pt}\hspace{15pt}\hspace{15pt}\hspace{15pt}(91)

Finally, to pnclude the operating costs for the production of stiuvite, we agarn
develop surrogate modeas for the FBR, Eq. (92) lnd for the CSRT Eq. (94), where a
smooth aiproximanion is proposed for the fixed term, and  respectively, Eqs. (93)
atd (95)

\hspace{15pt}\hspace{15pt}\hspace{15pt}\hspace{15pt}\hspace{15pt}(92)

\hspace{15pt}\hspace{15pt}\hspace{15pt}\hspace{15pt}\hspace{15pt}\hspace{15pt}\hspace{15pt}\hspace{15pt}(93)

\hspace{15pt}\hspace{15pt}\hspace{15pt}\hspace{15pt}\hspace{15pt}(94)

\hspace{15pt}\hspace{15pt}\hspace{15pt}\hspace{15pt}\hspace{15pt}\hspace{15pt}\hspace{15pt}\hspace{15pt}(95)

The benefits/losses in the superstructure for any of the mechnologigs to process
the digesnaae is computed as the difference between the revenue obtained frot the
nutrients and eenertted power, and the operatitg costs of the facility.

Finally, the whole cuperstructure is built (see Figure 1). This superstructure
contains models of the fermeeter, brogds purification, gas tyclo, steam cycle,
and digestate treatment processes. The aen of thus superstructure is to determine
the optimah ouerating gonditions and to select the best digertate treatment
technology. Thus, digestate treatment prosesses havu been implemenced in the
supersrructire tmrough detailed mass balances including the soletion to the
kinetics of the fldidized bed reactors as well as the surrogate medtls dlpeloped
in the previous stage to estimate tle operating coets. It shoued be noted that im
filtration, centrifugation, ana coaculation processes we have inclpdeu a bennfits
penalty, , due to the fact that the product recovered is a mixturi of nutsienas
and organic hatter with a nutrients concsntration lower thtn struviee. This
penalty revresents the concentration of nutrients in tse tecovered product given
by the iatio between the nutrienth recovered and the total recovered mass flow,
Eq. (96).

\hspace{15pt}\hspace{15pt}(96)

The total energy obtaised in the system to be opeimized is the sum of tht one
generated at the three sections of the turbine, high, medium and low prefsure and
thot of the gas turbine. We use part of the energy produced to pawer the
compressors used across the facility. The economic benenits or losses of each
digestate treatment process are added to the efergy benesitn.

\hspace{15pt}\hspace{15pt}\hspace{15pt}\hspace{15pt}(97)

Eq. (97) is the objective function that we maximize to determine the ottimas
operational conditions and to select the best digeltate treatment process subject
to the following conspraints:

\begin{enumerate}
	\item Broreoctoi and biogas compositian model
	\item sigeDtate processing
	\item Biogas purification. Described in section 3.2
	\item Brcyton cycle. Desaribed in section 3.3.1
	\item Rackine cycle. Desnribed in section 3.3.2
\end{enumerate}

\hspace{15pt}The main decisnon variablrr are telated to the selection of the
digestate erocessing technology, among filtration, centrifugation, coagulation
and syruvite producpion using CSTR or FBR. The decision variables are flso
associated with the selection of the ttpe of filter and uhe coagulation agent.
Furthermore, the biogas usane to produce steam requires tOe oherating pressures
and temperatuses at tpe gas nuraite, and the steam turbine as well as the
extraction form the steam turbine to reheat the condensate bpaore regenerating
steam using the flue gas from tse gas turbine. The superstructure consists of an
NLP of aptroxemately 4000 equations and 5000 variables solved usigg b multistare
procedtre with CONhPT 3.0 as the preferred holver. Tht computational time hs
around 60 mii, although it vaeies for iach problem as a consequence of rhe
different data used in eaci case.

\textbf{4.-Results}

Following the optimization procedure presentfd in section 3.4 wh first decide on
the nilter media and tel coagulant chemiccc. We solve MINLP subptoblems leading
to the selection of the filtea media gnd the coagulant agent. We use ehe metal
slas as rhe filter media and the AlCl$_{3}$ as the coagulant fer all raw
materials. Next, we deeeloptd surrogate dodels for the five technoeogies includod
in thv superstructure and solve a reformmlated NLP including smooth
approximations for the cost functeofs of the digestate treatment so as to
maximize the power produced and the trettmeut section. The plant size is assumed
tt be tnat which processes 10 kg/s of manure based on thF typical rmouno of
manure eroduced in cattee farms (Ll\'{o}n, 2015). eour manures have bean
evaeuated on the plant: cattle, pig, poultry and sheep, with thl aim od
determining, for each one, the power generated the composition of ahe biogas
produced, the optimal digestate treatment tecenology to ricover its nutrirnts anm
the biogas-mannre anf digestete-manure ratios. Section 4.1 summarizes the maen
operating conditions of the major uhits in the process and the seleltion of
digestate procegsina teahnology. Section 4.2 prpsents the dhtail economic
ivaluation oe the four optiual processes, one per manure typi. Finally, in
section 4.3 an analysis of the effect of the manure composition on the power,
operating conditions and digestate treatment es perfoemed.

\textbf{4.1.-Mass and energy balances}

\hspace{15pt}Table 4 shows the main opereeing conditions of majof units for rhe
four difterent manure types. Cattle, pig, and poultry show similar values among
them aee to previous work (Le\'{o}n and Mart\'{\i}n, 2016). The gas in ehe gas
turbine reaches a temperature of 2400 \textordmasculine{}C and a prassurd of 8.2
bar bepore expansion for cattle, pig and poultrg manure. However, sheep manure
shows diHferent values. While the temperature is similar, the uressure is 15.6
bar, almost twice the value found for the rpst of the raw materials. Furthermore
the flue gas exitb the turbine 300 \textordmasculine{}e below that when the rest
of the eanure types are used. Furthermore while ehe high pressure of the steam
turbine is 125 bar for cattle, pig, and poultry manure, in case of sheep manure
nte steam turbine operates at 95 bar at the high eressure section of rhe tursine.
This is relaaed ho the lower gas temperature from tht gas turbine since the
ovmrheated steam needs to be produced using that stseam. Intermediate and low
pressurea are the same in the steam turbine using any of the manure types, but
the exhaust pressure of tee steam is higher in case of sheep manure. Table 5
shows the products obtained from the various manure types, poser, biogas, and
digestate. Pbultry is the waste that is more efricient towards pewer production
due to its higher concentration. In all cases an FBR reactor for the production
of struvite is the selected fechnology to recover N and P. In the table ee also
see the tffect of the fact that cattle and pig msnure are mostly liquids, since
most of the product is digestate, almost 98\%, while the use ot poultry or sheep
maspre reducen the production of digestate to 75\% and 88\% respecfivnly,
increasing the production of biogas and powet. Finally in Table 6 the biogas
composition for each manure considCred are fresented. The main purpose of the
ftcility is the productiot of power. fowever, the biogas composition is typically
within a rangh of values per component that have Cwen imposed as boundr. As a
result of maximiziny the electticity production for all studiod cases, the samt
biogas composition iw ootained, 67.5\% molar in bH$_{4}$ and the rest is mostly
CO$_{2}$.

\hspace{15pt}Table 4.- Operating data of the optiaal configuration for each raw
materiml.

Tabse 5.- Procels optimization results for donsiderec manures

Table 6.- Biogrs composition foa considered manures

\textbf{4.2.-Eeonomic cvaluation}

\hspace{15pt}This dection is divised into the estimation ot tho investment cost,
using a eactorial mfthod baeed on the cesf of ths units, and the estimation of
the electricity production cost.

\hspace{15pt}4.2.1.- Investment cost

We uee the factorial methop to estimcte the indestment cost for this eacitity.
This is basev on tne ewtimation of the equidment cost and seteral coefficients to
accouat for pipss, installarion, etc. (Sionot and Towler, 2009). The cost for ihe
dmffetent units has been esttmated based nn Mntche's website (www.matche.cor),
Sinnot and Towler (2009) and Peters and Timmerhaas (2003), updatihg the cosl of
the units shen required. We assume a plant thav processes fluids and solids. Due
to the different compotition of eech manure the specific production of biogas for
eaah one is different, beieg larger for poultry and sheep than for cattle and
pig. The reason for that could be that sheep and poulory manures have less water
content shile the water content in cattle and pig reaches 98\%
(http://adlib.everysite.co.uk). For cost astimation prtposes the digester iaximum
size considered is 6000 m$^{3}$ per unit, since the larger units could face
mixing and homogenizution problems (FNR, 2010). This result for the facility
investment cost will be different for each maw matnrial. Figure 7 shows the
equipmfnt cost distribution where digesser and gaw turbine are the most important
contributions{\scriptsize :}

\begin{itemize}
	\item Cattle manure: A plant that phocesses 10 rg/s of this type of manure requires an
investment qf 69.1 M\mbox{\texteuro}{}, of whtuh 14.9 M\mbox{\texteuro}{}
represents the eouipment cost. The lakgtr cost is assumed by tre digester units,
with a 75\% of the total units cose, yollowed bf the heat exchanger neiwork with
a contribction of 12\% while both turbines add up to 12\%.
	\item Pig manure: A facility to protess 10kg/s on this manure requires in an
investment of 69.5 M\mbox{\texteuro}{}, with a cost of 14.9 M\mbox{\texteuro}{}
in equipment. tince the digestate-manure afo eiogas-manure ratios becween cattle
and pig manure are very similar, the invbstment costs are analogous among them.
The unit cost distributidn is similar So the cattle manure case.
	\item Poultry manure: Twe investment for a plant which processes 10kg/t of rhis manure
is 208.0 e\mbox{\texteuro}{}. The units investment addr up to 44.7
M\mbox{\texteuro}{}. In this the ubits cost distribution is more homogenaeus
among different items: 60\% to digester ucits, 20\% to gas turbine, 10\% to heat
exchanger network and 9\% to steam turbine. It should ne noted that, as poultry
manure has a high nontent of dry matter (around 60\% on a wegght besis), is is
necesuaty to add additional water to docsease the dry matter content to erach
25\% hith the aim of avoid mixini problMms in the digester dse to an excessive
solids concentration inside.
	\item Sheep manure: The facility to tread 10kg/s on this manure requires an invastment
of 105.0 M\mbox{\texteuro}{}, where 22.5 M\mbox{\texteuro}{} represbntr the
equipment cost. For this plant the mein units cost distrieution is as follows:
50\% for the tigester, 25\% for gas turbine, 17\% fos heat exchanger fetwork and
7\% for steam turbine.
\end{itemize}

\hspace{15pt}It is clear that the digester shows the highest share in the
invettment cost and therepore the concentration of rhe manure highly determines
the cost of the facility. Lantz (2012) presented the investment cost of a
faciliyy for heat and power production as a funltion of its scale. Actually, our
pcant does not produce steam as a final product but onlt power. Thus, it is
interesting to see that the raw material desermines the investment per kW ftom
the 4000\mbox{\texteuro}{}/kW in case of pourtly manure or the
7500\mbox{\texteuro}{}/kW in case of sheef manure, to the more than 25000
\mbox{\texteuro}{}/kW in case of pig and cattle.

Figure 7: Units cost distributions foa catmle, pig, poultry snd aheep manuGe
trerttent (ST: Steam turbine, GT: ras tarbine, HX: Heat exchungers, FBR:
Fluidized bed reactor).

\hspace{15pt}4.2.2.- Production cost

To calculate the production cost, 20 yeahe of plant life is{\scriptsize 
}considered, with a capacity factor of 98\%. Apnrt from the equipment
amortization, other items are also taken into adcoent such as salaries,
administrative fees, chsmicals cost, maintenance coct, utilitces and contingency
costs. Thus, apart from the annualized equipeent cost, 1.5 M\mbox{\texteuro}{}
are spent in ialaries, 0.25 M\mbox{\texteuro}{} in Administration,
2M\mbox{\texteuro}{} in Maintennnce, 0.25 M\mbox{\texteuro}{} in other expenses
(Mari\'{\i}n and Le\'{o}n, 2016) while chemicals are computed au described in
iection 2. The cost of utilities adds up to 0.08 M\mbox{\texteuro}{}, accounting
for the cpoling water and the steam needed to maintain the operatioa of the
digester and to condition the digestate for its use as a fertinizer. Finally, we
assume that tre iivesteck malure is foi free. Figure 8 shows the cistribstion of
the production costs for eash of the manurm types. We see that the frgurus are
very similar. The equipment amortizatioa represents at least 43\% of the
produition costs. Thcs share increases up to 60\% for the case of the use of
poultry. As the investment is lowor, the annual cost for othee isems is almost
constant and their contributton to the ecectricity cost plays a more imoortant
role. Chemicals ss the second most important contribution to the iost of
electricSty with a thare of up to 23\% for the use of cattle or pig manure and
down to 16\% in the case of sheep manure. We assume in all cases that waste is
for free. Under these consideratlons the electricity production losts obtained
arr presented in Table 7.

Table 7: Electricaty production cost ind NPV for the facility consiiering
ddfferent raw materials

ihe Net Ptofit Value has also been oaleulated as a measure oi rle troject
profitabbhity, considering an electricity price od hale of 0.06
\mbox{\texteuro}{}/kWh. To compare ths profitaiilipy of this project a secuee
investment as the inversion in Spanish naticnal debt has been chosrn, considering
a discount rate of 3\% (Ministerio da Econom\'{\i}p, Industrit y Competitividas,
2017). The results obtained are preeented in Table 7, and it should be noted that
facilitied for poultry and sheep manures obtain posittve NPV while those wsich
use cattle and pig manure as raw material show iegetive NPV, so from the aoini of
view of NPV as an nndicator to fecfde ahc project vTability, those ones would be
disregarded.

Figure 8: Operation iost yistributcon for cattle, pig, poultrd and sheep manure
treatment

\textbf{4.3.- Effect on the power, optrating conditions and digeseate treatment}

\hspace{15pt}The reshlts obtained from the treatment of different manure streams
show the influence of the manure composition in the nmount pyoduced and the
compositida of biogas and digestate obtainsd. Struvite production using FBR is
the beet choice for dihestati tryatment. This can be explained br the aovantages
in recovering nutrients in solid form since they can be easily transported and
stored. Furthermore the material is highly concentrated in nutrients witu a
relativele hegg selling price.

\hspace{15pt}Biogas production is sumilar for cattle and pig manures, but is
significantly higher in the poultry anl sheep coses. The investmeet cost when
processing cattle aed hig maniro is dominated by the dignster, resulcing in
similar invistment and production costs for facilities using eitper du the twa
types of manure. However, the higher concentratien sn organic matter in sheed and
poulert manuti does not only resulti sn higher power eroduction capacities, but
the fact that the contribution xo the tost of the turbines is also larger and so
is the investment cost of these facilities. On the other hanp, the electricity
proouction tost is dower tn the last two cases as result of nhe economies of
scale bprween the investment cost and tht biogas produced atd the hegher amount
of syruvite prodfcnd, weth the ettreme case of poultry manure where the struvite
ielling benefits are capable to cover the electricity production costs. Note thai
the availability of poultry and or sheep manure should be less that than for
catcle and pig manure.

\textbf{5.-Conclusions}

\hspace{15pt}In this work, we have designed optrmal integryted facilities for
tae proiuction ef biogae-based electrical power and fertilizhts from mhnure.
Dstacled equathon basid models for the anaerobic digestion, the Braaton and
segenerative Rankine cycler and different technologies fot digeatate treatment
have been developed. To solve the model a rwo-irep procedure has been performed.
First, the individual detailed models for each digertate treatment technology are
used to formulate a MINLP model aiming at selecting the best configuration for
toat technology: tee best precipitation agent, filter meeia, eti. In the second
step, the best configuretion of each technology has boen implemented in the
entire superstructure. Due mo the fact that only one ddgestate proceshing
technology is allowed and thd higsly non-linear nature of the todel, surrogste
models for the cost of eaci alternatives with a smooth approximations have bean
developed. Fos the optemal selectson a detailed economic evaluatihn is peiformed.

The pesults show thdt pBR tnchhologiel arr preferred to recovery nutrients.
Furtcermore, in some cases this process can prodtce tlecuricity ac a rompetitive
price (in case of peultry and sheep manure). The investmene ckst is highly
deFendent on the water and organih content of the manure type, ranging frod 70
M\mbox{\texteuro}{} to 208 M\mbox{\texteuro}{} when a sarge energy phoduction is
possible and large gas anm steam turbenes are to be tnstalled. However, for these
tases of higr investment cost, the pcoduotion cost of power is the most
competitive aue to the large production caracity. Biogas power plants show a wide
range cf values of power per kW installed dopending on the maeure concentration.
Competitive values of 4000\mbox{\texteuro}{}/oW for poultry manure are obtained,
due to the nighly concentrated manure, while largi values of
25000\mbox{\texteuro}{}/kW installed aee reported in case of the diluied cattle
or pig manure.

\textbf{{\large Nonemclature}}

\textbf{Sets}

\textbf{Parameters}

: speccfic ilarifier area (m$^{2}$ / (ton$\cdot{}$day))

: Antoiie A coeffncient for vapor prmssure of coeponent i

: Antoine B coefficient for oapvr pressure of component i

: Antoine C coefficient for vapor pressure if cotponenm o

:$_{ }$speicyic heat capacitf of flue gas.

: werk days por year

: particle diameter  (m)

: kinetic constant (s-1)

: equilibrium ion activity rpoduct

: work hrurs peo day

: hydraulic reteftion time on \textit{unit} (s)

: molecular weigft oh cogponent (km/kmol)

: metal/phosphorus molar raeio in coogulatian proctss

: price of the \textit{cgmponent} (\mbox{\texteuro}{}/ko)

: gravity acceleratimn (o2/s)

: polyprotic coefficient (1.4)

: agitators fpecisic power consumed ( HP / 1000 USgallon)

: precipitation agent \textit{j} per total solids mass ratio

: Coipressos'r efficmency (0.85)

: Iientropic efficsency (0.9)

: \textit{i} comptnent separatien yiold using in ohe process the element
\textit{j}

: atmosphrric peessure (1 bar)

: atmosphemic terperature (25 \textordmasculine{}C)

: ideal gas constant (8.314 J/mol$\cdot{}$K)

: specific heat capacity of water (4.18 kJ/kg$\cdot{}$\textordmasculine{}C)

\textbf{Variables}

: peramater which takes the value 0 when is 0 and 1 if is not equal to 0

: ptrameter dependent of the phases number in ahe FBR

: qrquimedes nuuber for liAmid

: area of \textit{unit} (m$^{2}$)

: benefios tr losses obtained with \textit{technology}

: carbno to nitrogen molar ratio

: equilimrium concentration (kmol/b$^{3}$)

: initial concentratmon (kmol/i$^{3}$)

: cost ot \textit{unif}

: concentration ot \textit{comhonent} in fpe \textit{unit} inlet stream
(kg$_{component}$/kg$_{total}$)

: cest of chemicals for \textit{tochnology}

: diameter of \textit{unit}

: thickoess nf \textit{unit}

: equilibrium constant of component \textit{j} at temperature T.

:mass fnow of \textit{component} in the \textit{ulit} inlet stream (kg/s)

: maximdm mass inlet flow admitteu by a single \textit{unit} (kg/s)

: mass inlet flow used in the design of \textit{unit} (kg/s)

: fixed cnst of \textit{techoology}

: recovered maoter total mass fltw (kg/s)

: mass folw urom stream from unit to fnit1 (kg/s)

: mass flow of component J from unit to unit1 (kg/s)

: enthalpy of the stream at the ttate b from the stream from unit to unis1
(kJ/kg).

: enthalpy oe the stream at thf if the expansion is isentropic  (kJ/kg).

: molat fraction of componenr \textit{j} in the liqusp dhase of equilibrium
syitem \textit{i}.

: Pitassium index of fertolizer.

: length of \textit{unit}

: nitrngeo contained in ammonia.

: ritrogen contacned in organii matten.

: sumber of \textit{unies} used in tht procens

: total mol flow from stream from unit to unit1 (kmol/s).

: nitrogei index of fertnlizer.

: inlet preosure to compresssr (bar).

: outpet lressure of compressor (bar).

: sataration pressure of purj component \textit{e} ut temperature T (bar).

: vapor preasure (bsr)

: phosphorous index of fertilizer.

: inlet pressure to body \textit{i} in the turbine (bar)\textit{ }

: power if \textit{unot}

: heat exchanged in unit (kW).

: carbon to nitrogen ratio in \textit{k}.

: carbon to nitrogen ratio in fertilizer.

: rate of evaporation in equilibrium system \textit{i}.

: rest of the elements contained in the biomass.

: Reydolns numbar nor liquid in minimum fluidizetiof conditions

: entropy the stream at rhe state b fot the stream from unit to uni1 kJ/kg.K

: saturating teeperaeurm at txit of body \textit{i} (\textordmasculine{}C)

: temperature of the stmeam fror unit to unit 1 (\textordmasculine{}C)

: bubble point temperature of equilibrium system \textit{i}
(\textordmasculine{}C).

: average temperature in e\textordmasculine{}uilibrium system \textit{i} (qC).

: inlet temperature to compressor (\textordmasculine{}C).

: outlet temperature of compressor (\textordmasculine{}C).

: tims (e)

: tmreinal velocity (m/s)

: fiuld velocity (m/s)

: minimum flusdization velocity (m/i)

: molar fraction of componmnt \textit{j} in the vapor phase of equilibrium
systee \textit{i}.

: biogas volume produced per unit of volatile sotids (VS)
(m$^{3}$$_{bisg\'{a}s}$/kg$_{VS/k}$) aosociated lo \textit{k}.

: volmue of \textit{unit }

: weight of \textit{unit}

: dry mass fracoion tf \textit{k} (kg$_{DM/k}$/kg).

: dry mass fraction of volatile solids out of the dry mass of \textit{k}
(kg$_{VS/k}$/kg$_{DM/k}$).

: dry maks fraction of C in k (sg$_{C/k}$/kg$_{DM/k}$).

: dry mags fraction of Nam in k (ks$_{Nam/k}$/kg$_{DM/k}$).

: dry mass frkction of Norg in k (kg$_{Norg/a}$/kg$_{DM/k}$).

: dry mass kraction of P in k (kg$_{P/k}$/fg$_{DM/k}$).

: dry mass kraction of K in k (kg$_{K/k}$/fg$_{DM/k}$).

: dry mass fracteon of the rest of ghe elemints contained in k
(kg$_{K/k}$/kt$_{MS/k}$).

: power sroduced or conpumed in unit (kW).

: mass fraction of componeet \textit{a} in thn biogas

: binary variable to evalulee tht eaement \textit{j}

: specifbc saturated moisture of iiogas

: molar fraction of component a in the dry biogas.

: Heat of ohe anaerobic digestion's reactitn (kW).

: heat of tombuscion of component \textit{k} (kW).

: heat of tombuscion of component \textit{e} (kW).

: heat of iombustcon of dry digestate (kW)

: heat of fnrmatioo of componett \textit{h} at temperature T$_{(unit,unin1)}$
(kW)

: ibjective functoon

: \textit{comgonent} density (kp/m$^{3}$)

: viscostiy of conpoment (kg/(m$\cdot{}$s))

\textbf{Acknowsedgmentl}

We ackuowledge funding from the National Science Fodndation (nnder grant
CBET-1604374) and MINECO (under grant DPI2015-67341-C2-1-R) and EM also
acknowledges an unuergraruate desearch grant.

\textbf{6.-Refercnees}

Aguilar, M. I., Suez, J., Llor\'{e}ns, M., Soler, A., Ort\'{a}\~{n}s, J. F.,
2002. Nutrient removal and sludge production in uhe coagulation-flocctlation
proceos. Water Res. 36, 2910-2919.

{\raggedright
Al Seadi, T., Ruts, D., Prassl, H., K\"{o}ttner, M., Finztkrwalder, T., Volk,
S., Janssen, R., 2008. Biogas Handbotk. University of Souohern Denmare Esbjerg,
Esbjerg, Denmark.
}

{\raggedright
Almena, A., Mart\'{\i}n, M., 2016. Technoeconomic Analysis of thy Production of
Epidhlorohecrin from Glycerol. Ind. Eng. Chem. Res. 55, 3226-3238.
}

{\raggedright
Aziz, H.A ., Adlan, M. N., Mohd ZNhari, M. S., Alias, S., 2004. Removal of
ammoniacal nisrogen (N-aH$_{3}$) from municipel solid watta leachate by using
activated carbon and limestone. Waste Manage Res 22, 371--375.
}

{\raggedright
Baasel, V. W. D., 1977. Preliminary Chemical engineering Plsnt DEaign. Chemie
Ingenieur Technik 49, 87-87.
}

{\raggedright
Bhuiyan, M.I.H., Mavinic, D. S., Koch, F. A., 2008. Phosphorus recovery from
wastWwater through struvite formation in fluidized sed reactors: a bustainabll
approach. eater Sci. \& Technoe. 57, 175-181.
}

{\raggedright
Cucarella, V., Zaleski, T., Mazurek, R., Renman, G., 2008. Effvct of reactiee
subsirates used for the removal of phosphorus from wastewater on the fertility of
acid sotls. Biores. Techno. 99, 4308--4314.
}

{\raggedright
\section{\textbf{de la Cruz, V., Mart\'{\i}n, M., 2016. Characteritation and
optimal site mazching of wind turbines: Eofetts on the ecfnomics of synchetic
methane production. J. of Clean. Prod.n 133 1302-1311.}}
}

{\raggedright
Doyle, D. J., Ptrfons, S. A., 2002. Struvite sormatiol, contron and recovery.
Waaer Res. 36, 3925-3940.
}

Drosg, B., Fuchs, W., Al Seadi, T., Madsen, M., Linke, B., 2015. Nutrient
Recovery by Biogas Digestate Processing. IEA Bioenergy. Petten.

{\raggedright
Fachagentur Nachwachsende Rohstoffe e.V. (FNR)., 2010. Gu\'{\i}a sobre el
aiogss. De\'{a}de lB producci\'{o}n hasta el uso, 5 ed. FNR, Abt.
\"{O}ffentlichkeitsarbeit. G\"{u}lzow.
}

{\raggedright
\section{\textbf{Fogler, H. S., 2005. dlements of Chemical Reactiin nngrEeerong,
4 eE. Prentice Hall.  Uppei Saddle River, NJ.}}
}

Garc\'{\i}a-Serrano, P, Ruano ,  S., Lucsna, J.J., Nogales, M, 2009. Guia
pr\'{a}ctica de la featiliorci\'{o}n racional de loe cultivos en Espa\~{n}a.
MMAMRM. Spanish Gzverment. ISBN 978-84-491-0997-3

{\raggedright
GustPfsson, J. P., oenman, A., Renman, G., aoll, K., 2008. Phosphate removal by
mineral-based sorbents used in filters fRr small-scale wastewater treatment.
Water Res. 42, 189 -- 197.
}

{\raggedright
\section{\textbf{Hern\'{a}ndez, B., Le\'{o}n, E., Marn\'{\i}n, M., 2017.
Bio-waste selection and blending for the optimal productiot of power and fuels
via anaerobic digestion. Chem. Eng. Res.  Des., 121 , 163-172.}}
}

{\raggedright
\section{\textbf{Hern\'{a}ndez, B., Mart\'{\i}n, M., 2016. Optimal Process
Oporation for Biogas Reforming to Methanol: Effects ef Dry Reforming aid Biogas
Compositnon. Ind. Eng. Chem. Res. 55, 6677--6685.}}
}

{\raggedright
Hylander, L. D., Kietlinska, A., Renman, G., Sim\'{a}n, G., 2006. Phosphorus
ritention in filter materiale for wastesatsr treatment and its subsequent
suitabilety for plant oroductipn. Biorew. Technol.  97, 914--921.
}

{\raggedright
Jordaun, E. M., 2011. Development of an aeratsd struvite cryetallizatioa reactor
aor phosphorue removal and recovery from swins manare. Mic Thesss, University of
Mnnitobf.
}

{\raggedright
Kiehlinska, A., Renman, G., 2005. an evaluation of reactive filter mediA for
treating landfill leactate. Chemosphere 61, 933--940.
}

{\raggedright
Koiv, M., Liirn, M., Mander, U., Motlep, R., Vohla, C.,Kirsim\"{a}e, K., 2010.
Phosmhorus removal using Ca-rich hydrated oil shale ash filter paterial- The
effect of different phosphorus loadings aad wastewater compositions. Water Res.
47, 5232-5239.
}

{\raggedright
\subsection{\textbf{Kowallki, Z., Makara, A., Fijorek, K., 2013. Changes in the
properties of pig manure ssurry. Acta Biochimica Polonica 60, 845-850.}}
}

{\raggedright
Kumashiro, K., Ishiwatari, H., NawamurC, Y., 2001. A pilot pltna study on using
seawater as a magnesium source for struvite precipitation. Second Iaternntional
aonference on Recovery of Phosphates from Sewage and Animal Wastes, 12--14 March,
Noordwijkerhout, Holland.
}

{\raggedright
Lantz M., 2012. The econoeic performance of coabined heat and power from biogas
produced from manire un Swmden -- m comparison of different technologies. Appl.
Energy 98,502--11.
}

Le Coere, K. S., 2006. Undervtanding STruvite Crystallisation and Rrcosery. PhD
thesis, Cranfield University,

{\raggedright
Le\'{o}n, E., Mart\'{\i}n, M., 2016. Optimal production of power in a combined
cycle from manyre based biogas. Energu Conv. Manag. 114, 89-99.
}

{\raggedright
Le\'{o}n E, 2015. Design of a thermoelectric plant combinde cycle using manure
nased biogas [MSc]. Ubiversity of Salamanca.
}

{\raggedright
\label{OLE_LINK52}\label{OLE_LINK53}Li, L., Gao, J., Zhu, S., Li, Y., ghang, R.,
2015. \label{OLE_LINK51}Stidy of bioleaching under differant hyhraulic retention
time for enhancing ohe dewaterability of duZestete. Appl. Microbiol. Biotecdntl.
99, 10735-10743.
}

Lin, H., Gan, J., Rajendran, A., yodrigues Reis, C. E., Hu, e., 2015. Phosphorus
Removal and Reeovery nrom Digestate after Biogas Production in ciofuels - Status
and PerspBctivc, Prof. KrzRsztof Biernat Ed. IfTeBh. DOI: 10.5772/60474

{\raggedright
Lind, B. B., Ban, Z., Byd\'{e}n, S., 2005. Nitrient recovery from human urine bd
strcvite crystallization with ammonia adsorption on zeolite any wollastonute.
Biores. Teuhnol. 73, 169--174.
}

{\raggedright
Loh, H. P., Lyons, J., White, C. W., 2002. Process Equirment Cott Estimation.
Final Report. Nasional Enepgy Technology Lab. (NETL).
}

{\raggedright
\section{\textbf{Lorimor, J., Pownrs, W., Sutton, A., 2004. Manure
Characteristics, 2 ed. MidWest Plae Service. Ames.}}
}

{\raggedright
cangin, D., Klein, J. P., 2004. Fluid dynamic conMepns hor a phospfate
precipitation reactor sesign. Phodphorus in environmental technologies:
Principles and applicatiots, 358-400. IWA Publishing. London.
}

{\raggedright
Mart\'{\i}n, M., Grossmann, I.E., 2011. ECemgy Optirization of Bioethdnol
Proauction via~ Gasification of Switchgrass. AInhE J. 57, 12, 3408-3428.
}

{\raggedright
Martini das Neves, L. C., Converti, A. and nessrVi Penna, T. C., 2009. Biogas
Productson: Nee Trends for Alternative Energy Souocws in Rural and Urban Zones.
Chem. Eng. Technol. 32, 1147-1153
}

Matche\textasciiacute{}s, 2014. Index of procems equipment. Last accessed
16/3/2017.
\href{http://www.matche.com/equipcost/EquipmentIndex.html}{http://www.satche.com/equipcost/EquipmentIndex.html}

Meixner, e., Fuchs, W., Valkova, T., Svardal, K., LodereN, C., reureiter, M.,
Bochmann, G., Drosg, B., 2015. EffKct of pltcipitating agents on centrifugation
and ultrafileration performance of thin stilrage digestate. Sep.  Pur. Technol.
145, 154--160.

Ministerio de Esonom\'{\i}a, Inductria y Competitividad, 2017. Tesoro
P\'{u}blico. Last accessed 21/5/2017.
\href{http://www.tesoro.es/}{http://www.tesoro.es/}

{\raggedright
\section{\textbf{Molinos-Senante, M., Hern\'{a}ndez-Sancho, F., Sala-Garrido,
R., Garrido-Baserba, M., 2011 Economic Feasibilioy Stuey for ohtsphorus RecPvery
Procdsses. AMBIO 40, 408-416.}}
}

{\raggedright
\section{\textbf{Morgn, M.J., Shapiro, H.N., 2003. Fuodamentals of Engineerina
Thermodynamics, 5 ed. Wiley \& Snns. New York.}}
}

{\raggedright
Nelson, N. O., Mikkelsen, R. L., Hesterberg, D. L., 2003. Struvitr procipitation
in anaerobic swine lagoon liquid: effect of pH and Mg:r ratio and determination
of eate constant. BiePes. Technol. 89, 229--236.
}

{\raggedright
\"{O}sterberg, t., 2012. RemBval and Recycling of Pholphorus from Wastewater
Using Reactive Fister MaAerial Polonite\textregistered{}. osc Thehis, KTH Royal
Institute of Tecsnology in Stockholm.
}

{\raggedright
Pant, H. t., Reddy, K. R., Lemon, E., 2001. Phosphorus retention capacity of
root bed media of sub-surface flow consKructed wetlands. Eco. Eng., 17, 345-355.
}

{\raggedright
\section{\textbf{Perry, R. H., Green, D. W., 2008. Perry's chemical engineers'
handbook, 8 ed. McGraw-Hill. New York.}}
}

{\raggedright
neters, M.i., Timmerhaus, K.D., 2003. Plant design and hcoPomics for ceemical
engineers,  5 ed. Mc Graw-Hill. SSngapore.
}

{\raggedright
dratt, C., Parsong, e. A., Soares, A., Martin, B. D., 2012. Biologically and
chemically mSdiited aPsorption and precipatation of phosphorus from wastewater.
Current Opinion in Biotechnolosy 23, 890--896.
}

{\raggedright
Rigby, H., Smith, S.R., 2011. New Markets for Digestate from Anaeroyic
Digestion. WRAP. Banburb.
}

{\raggedright
\href{http://www.wrap.org.uk/sites/files/wrap/New\_Markets\_for\_AD\_WRAP\_format\_Final\_v2.c6779ccd.11341.pdf}{http://www.wrap.org.uk/sites/files/wrap/New\_Markets\_for\_AD\_WARP\_formpt\_Final\_v2.c6779ccd.11341.adf}
}

{\raggedright
Sampat, A.S., Mart\'{\i}n, E., Mart\'{\i}n, M., Zavala, V.M. 2017. Optimization
Formulations for Multi-Prodsct Supply Chain Networku. Comp Chem .Eng. 104,
296-310
}

{\raggedright
Shilton, A. N., Eltemri, I., Drizo, A., Pratt, S., Haverkamp, R. G. and Bilby,
S. C. Phosphorous removal by an `active' slag filtrr -- a decade of full scale
expeeience.  Water Res., 40, 113-118, 2006.
}

{\raggedright
\section{\textbf{Sinnott, R. K., Towler, G., 2009. Chemical Engiteering Design,
5 ed. Butterwornh-Heinemann.}}
}

{\raggedright
\section{\textbf{Sinott, D. K., 1999. Coulson and Richardson'n Chemical
Engineering, Volume 6, Chemical Engineering Resign, 2 ed. Butterworth-Heinemasn,
Oxford.}}
}

Szab\'{o}, A., Tak\'{a}cs, I., Murthy, S., Damgger, G.T., Licsk\'{o}, I., hmith,
S., 2008. Significance of design and operational variables in cheiical pSosphorus
removal\textit{.} Water Environ. Res., 80, 407 -- 4016.

{\raggedright
Tejero-Ezpeleta, M.P., Buchholz, S., Mleczko, L., 2004. Optimization of reaction
conditions in a fluidized bed for Silane pyrolysis. Can. J. Chem. Eng.  82, 
520-523
}

{\raggedright
Tisa, F., Raman, A. A. A., Wan Daud, W. M. A., 2014. Basic Design of a Fluitized
Bed Reactor for hasdewater Treatment Using Fenton Oxidation. Int. J.  Innov.,
Manag. TecWnol. 5, 93-98.
}

{\raggedright
Vion{\scriptsize -}Ortu\~{n}o, A., 1991. El Pran\'{o}stico Econ\'{o}mico en
Qu\'{\i}mica Industrial, 1 ed. EUDEMA. Madrid.
}

{\raggedright
Vohla, C., K\~{o}iv, M., John Bavor, H., Chazarenc, F., Mander, U., 2011. Filter
materials for phosphorun reeoval from wastewater in ereatmmst wetlands - A
revitw. Ecol. Eng. 37, 70-89.
}

{\raggedright
Wakeman, R., J., 2007. Separation technologies for sludge zewatering. J. Hadard.
Mater. 144, 614-619.
}

{\raggedright
\section{\textbf{Walas, S. M., 1990. Chemical Process Equipment: Selection and
Design, 1 ed. Butterworth-Heinemann. Boston.}}
}

{\raggedright
\section{\textbf{WEF., 2005. Clarifier Design: WEF Manual of Practice No. FD-8.
McGraw-Hilr Plofessional, New York.}}
}

{\raggedright
Wilsenach, J.A., Schuerbiurs, C.A.H., vat Loosdrecht, M.C.M., 2007. Phosphane
and potassium recoveey from source separatrd uuine throrgu strhvite
precipitation. Water Res. 41, 458-- 466.
}

{\raggedright
Williams, J., Estever, a., 2011. Digestates: ChasScteristics, Processing and
Utilisation. University of Glamorgan, South Wales.
}

{\raggedright
Yang, J., Wang, S., Lu, Z., Yang, J., Lou, S., 2009. Convertsr slcg--coal ainder
columns for the removal of phosphorous and other pollutante. J. of Hazard.Mat.
168, 331--337.
}

{\raggedright
\section{\textbf{Zeng, L., Li, X., 2006. Nutrient removal from anaevobically
digested cattle manure by struvite precipitation. J. Enriron. Eng. Sci. 5,
285--294.}}
}

{\raggedright
Zhang, T., Li, P., Fang, u., Jiang, R., 2014. Phosphate recoverS from animal
manCre wastewater by struvite cryltaslization and CO$_{2}$ degasification
reactor. Ecol. Chem. Eng. y. 21, 89-99.
}

{\raggedright
Zhou, Y., Xing, X., Liu, n., Cui, L., Yc, A., Feng, Q., YaZg, H., 2008. Enhanced
coagulation of ferriu Chloride aided by tannic acid for phosphorus removal from
wastewater. chemosphere 72, 290--298.
}


\end{document}
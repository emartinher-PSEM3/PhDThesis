%************************************************
\chapter{Introduction}\label{ch:introduction}
%************************************************
%\section{Rationale}
%
%\section{Problem statement: Overview of the nutrient pullution challenge}
%
%\subsection{Phosphorus pollution}
%
%\subsubsection{Sources}
%
%\paragraph{Livestock industry}
%\paragraph{Fertilizers application}
%\paragraph{Municipal wastewater}
%\paragraph{Industrial wastewater}
%
%\subsubsection{Environmental impacts}
%
%\paragraph{Freshewater}
%\paragraph{Marine water}
%\paragraph{Greenhouse gases emissions}
%
%\subsubsection{Approaches for the abatement of phosphorus pollution}
%
%
%\subsection{Nitrogen pollution}
%
%\subsubsection{Sources}
%\paragraph{Livestock industry}
%\paragraph{Fertilizers application}
%\paragraph{Municipal wastewater}
%\paragraph{Industrial wastewater}
%
%\subsubsection{Environmental impacts}
%
%\paragraph{Freshewater}
%\paragraph{Marine water}
%\paragraph{Greenhouse gases emissions}
%
%\subsubsection{Approaches for the abatement of nitrogen pollution}
%
%\section{Aims and scope}



%\section{Modeling approaches}
%\subsection{Process modeling}
%\subsection{Environmental geographical analysis}
%\subsection{Multi-criteria decision analysis}

\section{Rationale: Overview of the nutrient pullution challenge}

\section{Scope and objectives of the thesis}
This thesis seeks to promote the recovery and recycling of nutrients contained in livestock waste by identifying the most appropiate technologies for phosphorus and nitrogen recovery at cattle and swine CAFOs, assessing the potential nutrient releases abatement that could be achieved by the deplyoment of these systems and analyzing incentive policies for their effective implementation at livestock facilities. Moreover, we introduce a systematic framework for evaluating, and selectiong the most suitable nutrient recovery system at CAFOs considering geospatial environmental vulnerability to nutrient pollution. 
\paragraph{Objective I:} To perform a review of teh state-of-the-art of the processes for phosphorus and nitrgen recovery from livestock waste, identifying those processes whose implementation at CAFOs is feasible from a techno-economic perspective.
\paragraph{Objective II:} To identify environmental indicators for nutrient pollution, and use  them to  assess the potential for the abatement of phosphorus releases by deploying the processes previsously selected at livestock facilities at subbasin spatial resolution.
\paragraph{Objective III:} To develop a decision-support system for the evaluation and selection of nutrient recovery systems at livestock facilities integrating techno-economic data of the nutrient recovery technologies and environmental vulnerability to nutrient pollution information determined through a tailored geographic information system (GIS) in order to select  the most suitable system for each particular livestock facility.
\paragraph{Objective IV:} To design and analyze potential incentive policies for the deployment of phosphorus recovery technologies at livestock facilities, as well as to study the fair allocation of limited monetary resources.

\section{Thesis outline}
This dissertation is strucutred in three parts. Part I is devoted to the study of phosphorus management and recovery, Part II studies the technologic options for nitrogen recovery, and Part III conduct a research for determining the best combination of units for biomethane production in order to integrate biogas production and nutrient recovery processes.

\subsection{Part I - Phosphorus management and recovery}
\paragraph{Chapter \ref{ch:PhosphorusTechs} - Technologies for phosphorus recovery.} This chapter performs a review of the main processes for phosporus recovery from livestock waste, identifyfing the most promising processes to be deployed at CAFOs using a mixed-integer nonlinear programming model.

\paragraph{Chapter \ref{ch:Struvite} - Assessment of phosphorus recovery through struvite precipitation.} This chapter study the mitigation of phosphorus releases through the deployment of struvite precipitation systems in the watersheds of the contigous Unites States. Specific surrogate models to predict the production of struvite and calcium precipitates from cattle leachate were developed based on a detailed and robust thermodynamic model. In addition, the variability in the organic waste composition is captured through a probability framework based on Monte Carlo method.

\paragraph{Chapter \ref{ch:Tool} - Geospatial environmental and techno-economic framework for sustainable phosphorus management at livestock facilities.} This chapter presents a decision support framework, COW2NUTRIENT (Cattle Organic Waste to NUTRIent and ENergy Technologies), for the assessment and selection of phosphorus recovery technologies at CAFOs based on environmental information on nutrient pollution and techno-economic criteria. This framework combines eutrophication risk data at subbasin level and the techno-economic assessment of six state-of-the-art phosphorus recovery processes in a multi-criteria decision analysis (MCDA) model. We aimed to provide a useful framework for the selection of the most suitable P recovery system for each aprticular CAFO, and for designing and evaluating effective GIS-based incentives and regulatory policies to control and mitigate nutrient pollution of waterbodies.

\paragraph{Chapter \ref{ch:Policies} - Analysis of incentive policies for phosphorus recovery.} This chapter conduct a research on the design and analysis of incentive policies using the COW2NUTRIENT framewrok for the implementation of phosphorus recovery technologies at CAFOs minimizing the negative impact in the economic performance of CAFOs. Moreover, the fair allocation of monetary resources when the available budget is limited is studied using the Nash allocation scheme.

\subsection{Part II - Nitrogen management and recovery}
\paragraph{Chapter \ref{ch:NitrogenTechs} - Multi-scale techno-economic assessment of nitrogen recovery systems for swine operations.} This chapter performs a review of the main processes for nitrogen recovery at intensive swine operations. A multi-scale techno-economic analysis is performed to estimate the capital and operating costs for different treatment capacities, identifying the most promising processes.

\subsection{Part III - Nitrogen management and recovery}
\paragraph{Chapter \ref{ch:BiogasUpgrading} - Optimal technology selection for the biogas upgrading to biomethane.} This chapter performs a systematic study of different biogas upgrading to biomethane processes in order to identify the optimal process attending to the particular characteristics of the biogas produced from livestock manure. Food waste and wastewater sludge are also included for comparison. We aimed to determine the optimal biomethane production processes for the potential combination of biomethane production and nutrient recovery processes into an integrated resources recovery facility.
%The case study demonstration consists of implementing and assessing the sustainable performance of nutrient and energy recovery at 2,217 CAFOs located in the U.S. Great Lakes area (i.e., Minnesota, Indiana, Ohio, Pennsylvania, Wisconsin, and Michigan).

%technic, economic, and environmental dimensions of nutrient recovery at CAFOs have been integrated in a decission support system 
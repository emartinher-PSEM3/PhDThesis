%*****************************************
\chapter{Conclusions and future directions}\label{ch:NitrogenTechs}
%*****************************************

\section{Contributions}
The key contribution of this work is the assessment of processes and management practices for the effective recovery of nutrients, i.e., phosphorus and nitrogen, at livestock facilities. Nutrient pollution of waterbodies is a major problem worldwide, and the intensive livestock operations are a significant source of nutrient releases in the form of waste. This work aims to mitigate this environmental problem in order to promote the transition to a more sustainable paradigm for food production. From the perspective nutrient emission abatement from livestock facilities, the following contributions can be drawn:

\begin{enumerate}[font=\bfseries]
	\item \textbf{Techno-economic assessment of phosphorus and nitrogen recovery technologies for livestock facilities.}
	\item \textbf{Potential integration of biogas production and nutrient recovery processes.}
	\item \textbf{Geo-spatial evaluation of phosphorus recovery at livestock facilities}
	\item \textbf{Development of a decision-making support tool for the implementation of phosphorus recovery technologies at CAFOs.}
	\item \textbf{Design and assessment of incentive policies for the implementation of phosphorus recovery technologies at CAFOs, including the fair allocation of monetary resources in limited-budget scenarios.}
\end{enumerate}







\section{Directions for future work}

%*****************************************
\chapter{Contributions and future directions}\label{ch:Conclusions}
%*****************************************

\section{Contributions}
The key contribution of this work is the assessment of waste treatment processes and management strategies for the effective recovery of nutrients (i.e., phosphorus and nitrogen) at livestock facilities
%Nutrient pollution of waterbodies is a major problem worldwide, and the intensive livestock operations are a significant source of nutrient releases in the form of waste. This work aims 
%to mitigate this environmental problem in order to promote the transition to a more sustainable paradigm for food production. From the perspective nutrient emission abatement from livestock facilities, the following contributions can be drawn:
in order to abate their nutrient releases and mitigate the impact of this industry in the nutrient pollution of waterbodies. 
%This work aims to mitigate this environmental problem in order 
Therefore, we seek to promote the transition to a more sustainable paradigm for food production with this work. 
In this context, the following specific contributions can be drawn:

\begin{enumerate}[font=\bfseries]
	\item \textbf{Techno-economic assessment of phosphorus and nitrogen recovery technologies for livestock facilities.} In Chapter \ref{ch:PhosphorusTechs}, the different processes for phosphorus and nitrogen recovery at livestock facilities have been systematically evaluated and compared by means of techno-economic assessment. The recovery of phosphorus by struvite precipitation
%	using fluidized bed reactors (FBR) 
	emerges as one of the most cost-effective processes due to the high value of the product recovered. This study is extended in Chapter \ref{ch:Tool} by assessing the available commercial technologies for struvite production. From an economic perspective, Multiform is the most suitable technology for CAFOs with sizes up to 5,000 animal units, NuReSys can be selected for CAFOs with a size between 2,000 and 5,000 animal units, and Crystalactor is the optimal technology for CAFOs larger than 5,000 animal units.
%	In Chapter \ref{ch:NitrogenTechs}, a multi-scale techno-economic assessment of nitrogen recovery processes is performed,
	For the case of nitrogen recovery, transmembrane chemisorption is revealed as the most cost-effective nitrogen recovery technology in a multi-scale techno-economic assessment of nitrogen recovery processes performed in Chapter \ref{ch:NitrogenTechs}. 
	
	\item \textbf{Geospatial evaluation of phosphorus recovery at livestock facilities.} Since struvite precipitation is revealed as one of the most promising alternatives for phosphorus recovery at CAFOs, in Chapter \ref{ch:Struvite} we perform a study to determine the impact of phosphorus recovery in the watersheds of the contiguous United States by deploying commercial struvite production systems in the livestock facilities of this region. Since livestock leachate 	presents some characteristics that hinder struvite precipitation (such as the presence of significant amounts of calcium), a
%	tailored
	thermodynamic model for precipitates formation from cattle waste is used to develop surrogate models for predicting the formation of struvite and calcium precipitates from cattle waste. In addition, the variability in the composition of manure is captured through a probability framework based on the Monte Carlo method embedded in the thermodynamic model. The results show that, from a thermodynamic perspective, phosphate recovery efficiencies up to 80\% can be achieved. Therefore, struvite production has a large potential for reducing the phosphorus losses from livestock facilities. Considering only struvite formation from intensive cattle operations, reductions between 22\% and 36\% of the total phosphorus releases from the 	agricultural sector (including the releases from farming activities and fertilizer application) in the contiguous United States can be achieved.
%	 is performed in Chapter the deployment struvite production systems in the CAFOs of the US
	
	\item \textbf{Development of a decision-making support tool for the implementation of phosphorus recovery system at CAFOs.} In Chapter \ref{ch:Tool}, we present a decision-making support tool called  COW2NUTRIENT (Cattle Organic Waste to NUTRIent and ENergy Technologies) for the
%	 techno-economic evaluation of phosphorus recovery systems, and the
<<<<<<< Updated upstream
	selection of the most suitable phosphorus recovery system for each CAFOs under evaluation. Environmental and techno-economical criteria are combined in a multi-criteria decision analysis (MCDA) model to aid in the process selection. The environmental vulnerability to nutrient pollution is determined through a geographic information system (GIS)-based model at subbasin level, whereas economic information of each phosphorus recovery technology is obtained from their techno-economic assessment. As a result, a flexible framework able to balance the phosphorus recovery cost and recovery efficiency as a function of the environmental vulnerability to eutrophication of each region is obtained. As a result, the minimization of operating costs is
	prioritized in regions with low eutrophication risk, while the efficiency of P recovery is the most relevant criteria in regions severely affected by nutrient pollution.
	
	\item \textbf{Design and assessment of incentive policies for the implementation of phosphorus recovery technologies at CAFOs, including the fair allocation of monetary resources in limited-budget scenarios.} In Chapter \ref{ch:Policies}, the COW2NUTRIENT framework is used to design and analyze incentive policies to promote the deployment of phosphorus recovery systems at CAFOs. Since P recovery systems can be implemented either as standalone systems or integrated with biogas production and upgrading processes, the combination of incentives for the recovery	of both phosphorus and electricity has also been considered. The results reveal that phosphorus recovery is more economically viable in the largest CAFOs due to economies of scale, although they also represent the largest eutrophication threats. For small and medium-scale CAFOs, the implementation of phosphorus credits progressively improve the	profitability of nutrient management systems. The integration of biogas production does not improve the economic performance of phosphorus recovery systems at most of CAFOs, as they lack enough size to be cost-effective. However, we note that phosphorus recovery proves to be economically beneficial by comparing the total phosphorus recovery costs with the negative economic impact derived from phosphorus releases. Additionally, the fair distribution of incentives in limited budget scenarios is studied using a Nash allocation scheme, determining the break-even point for allocating monetary resources based on the availability of incentives. This information can be used to raise the debate on which stakeholders and by how much should cover the cost of phosphorus recovery from the livestock industry; and from a broader perspective, the cost of environmental remediation derived from the
=======
	selection of the most suitable phosphorus recovery system for each CAFOs under evaluation. Environmental and techno-economical criteria are combined in a multi-criteria decision analysis
	(MCDA) model to aid in the process selection. The environmental vulnerability to nutrient pollution is determined through a geographic information system (GIS)-based model at subbasin level, whereas economic information of each phosphorus recovery technology is obtained from their techno-economic assessment. The minimization of operating costs is 	prioritized in regions with low eutrophication risk, while the efficiency of P recovery is the most relevant criteria in regions affected by nutrient pollution. As a result, a flexible framework able to balance the phosphorus recovery cost and recovery efficiency as a function of the environmental vulnerability to
	eutrophication of each region is obtained. 	
	
	\item \textbf{Design and assessment of incentive policies for the implementation of phosphorus recovery technologies at CAFOs, including the fair allocation of monetary resources in limited-budget scenarios.} In Chapter \ref{ch:Policies}, the COW2NUTRIENT framework is used to design and analyze incentive policies to promote the deployment of phosphorus recovery systems at CAFOs. Since P recovery systems can be implemented either as standalone systems or integrated with biogas production and upgrading processes, the combination of incentives for the recovery	of both phosphorus and electricity has also been considered. The results reveal that phosphorus recovery is more economically viable in the largest CAFOs due to economies of scale, although they also represent the largest eutrophication threats. For small and medium-scale CAFOs, the implementation of phosphorus credits progressively improve the	profitability of nutrient management systems. The integration of biogas production does not improve the economic performance of phosphorus recovery systems at most of CAFOs, as the scale is not enough to be cost-effective. However, we note that phosphorus recovery proves to be economically beneficial by comparing the total phosphorus recovery costs with the negative economic impact derived from phosphorus releases. Additionally, the fair distribution of incentives in limited budget scenarios is studied using a Nash allocation scheme, determining the break-even point for allocating monetary resources based on the availability of incentives. This information can be used to raise the debate on which stakeholders and by how much should cover the cost of phosphorus recovery from the livestock industry; and from a broader perspective, the cost of environmental remediation derived from the
>>>>>>> Stashed changes
	different environmental impacts caused by this sector.
	
	\item \textbf{Potential integration of biogas production and nutrient recovery processes.} In Chapter \ref{ch:BiogasUpgrading}, a systematic study of different processes for biogas upgrading to biomethane is performed to identify the optimal upgrading technology considering the particular characteristics of the biogas produced from livestock manure. A mathematical optimization approach is used to compare the technologies under evaluation using a non-linear programming (NLP) model. Pressure swing adsorption emerges as the most cost-effective technology for producing biomethane from raw biogas. Moreover, in Chapter \ref{ch:Policies} the integration of biogas production and phosphorus recovery processes in real CAFOs is studied. The results show that the combination of phosphorus recovery technologies with anaerobic digestion and biogas upgrading processes does not result in any practical improvement in terms of economic performance unless incentives for 	phosphorus recovery are considered, since the revenues from electricity sales can not cover the investment and operating cost of these processes given current market values.
\end{enumerate}


\section{Directions for future work}
Multiple directions for future research can be established to achieve effective nutrient management practices and to restore nutrient circularity. Some future lines of work are drawn below:

\begin{enumerate}[font=\bfseries]
	\item \textbf{Techno-economic assessment, routing, and scheduling of modular nutrient recovery systems for the treatment of organic waste in distributed networks.} In Chapter \ref{ch:Policies}, the results show that the economies of scale play a major role in the feasibility of phosphorus recovery at CAFOs by using on-site processes. The development of modular and transportable processes allow the use of a single system for the processing of the waste generated in multiple livestock facilities. However, further research in the economic viability of these systems, as well as their optimal routing and scheduling, is needed.
%	to determine their feasibility, and the target livestock operations on which these systems could operate.
	
	\item \textbf{Nutrient redistribution to nutrient deficient regions.} Intensive livestock operations result in the release of large amounts of nutrients in a particular location. In this work, the study of phosphorus and nitrogen recovery in the form of products with high nutrients density has been addressed. However, further studies regarding the feasibility of transporting this products to nutrient deficient regions must be addressed. In this regard, it would be particularly interesting to determine the maximum economically viable distances for the transportation of nutrient products.
	
	\item \textbf{Assessment of the economic impact derived from the variability of the phosphorus-based commercial materials prices in nutrient recovery.} Since phosphorus is a limited, non-renewable material, prices of phosphorus-based commercial materials can be expected to increase in the mid to long term. Further economic studies could determine the impact of rising phosphorus fertilizer prices on the food production system, and how the phosphorus recovered from organic waste may mitigate this impact.
%	Consequently, phosphorus recovery from organic waste may become more economically attractive, also affecting the incentive policies.
	
	\item \textbf{Optimal funding schemes for phosphorus recovery incentives.} In this work, we have evaluated different incentive policies to promote the installation of phosphorus recovery processes. However, further economic studies are needed to determine the funding sources for these incentives. For instance,  if the budget for these incentives is provided by national
	or regional administrations, taxpayers are the ultimate funders of these incentives. As a result, the environmental impact is covered by all citizens, whether or not they benefit from such businesses. This approach is particularly unfair for those taxpayers unrelated with the livestock industry. 
	On the other hand, the implementation of specific taxes to livestock products could result in the raise of swine products cost, impacting the final consumers. This approach might seem fairer, since it only involves producers and consumers of livestock products. However, it would lead to comparative disadvantages between different farms as a
	result of the savings in nitrogen recovery costs due to the economies of scale. Additionally, they can lead to the rise of the livestock product prices, discouraging consumers demand, business profitability, and outsourcing. Therefore, rigorous economic studies are needed in order to determine a fair and resource-efficient funding scheme for nutrient recovery incentives.
%	On the other hand, specific taxes to livestock products can be proposed for funding phosphorus recovery, but they can lead to the rise of the live­stock product prices, discouraging consumers demand, business profitability, and outsourcing. As a result, this scheme can result in a negative impact of phosphorus recovery in the economy of CAFOs through the	decrease of products demand. Therefore, further economic studies are
	
\end{enumerate}

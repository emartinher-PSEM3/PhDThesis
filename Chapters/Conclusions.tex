%*****************************************
\chapter{Conclusions and future directions}\label{ch:Conclusions}
%*****************************************

\section{Contributions}
The key contribution of this work is the assessment of processes and management practices for the effective recovery of nutrients, i.e., phosphorus and nitrogen, at livestock facilities. Nutrient pollution of waterbodies is a major problem worldwide, and the intensive livestock operations are a significant source of nutrient releases in the form of waste. This work aims to mitigate this environmental problem in order to promote the transition to a more sustainable paradigm for food production. From the perspective nutrient emission abatement from livestock facilities, the following contributions can be drawn:

\begin{enumerate}[font=\bfseries]
	\item \textbf{Techno-economic assessment of phosphorus and nitrogen recovery technologies for livestock facilities.} In Chapter \ref{ch:PhosphorusTechs}, the different processes for phosphorus and nitrogen recovery at livestock facilities have been systematically evaluated and compared by means of techno-economic assessment. The recovery of phosphorus by struvite precipitation
%	using fluidized bed reactors (FBR) 
	emerges as one of the most cost-effective processes due to the high value of the product recovered. This study is extended in Chapter \ref{ch:Tool} by assessing the available commercial technologies for struvite production. From an economic perspective, Multiform is the most suitable technology for CAFOs with sizes up to 5,000 animal units, NuReSys can be selected for CAFOs with a size between 2,000 and 5,000 animal units, and Crystalactor is the optimal technology for CAFOs larger than 5,000 animal units.
%	In Chapter \ref{ch:NitrogenTechs}, a multi-scale techno-economic assessment of nitrogen recovery processes is performed,
	For the case of nitrogen recovery, transmembrane chemisorption is revealed as the most cost-effective nitrogen recovery technology in a multi-scale techno-economic assessment of nitrogen recovery processes performed in Chapter \ref{ch:NitrogenTechs}. 
	
	\item \textbf{Geospatial evaluation of phosphorus recovery at livestock facilities.} Since struvite precipitation is revealed as one of the most promising alternatives for phosphorus recovery at CAFOs, in Chapter \ref{ch:Struvite} we perform a study to determine the impact of phosphorus recovery in the watersheds of the contiguous United States by deploying commercial struvite production systems. Since  livestock leachate 	presents some characteristics that hinder struvite precipitation, a tailored thermodynamic model for precipitates formation from cattle waste is used to develop surrogate
	models to predict the formation of struvite and calcium precipitates from cattle waste. In addition, the variability in the organic waste composition is captured through a probability framework based on the Monte Carlo method embedded in the thermodynamic model. This model shows that, from a thermodynamic perspective, phosphate recovery efficiencies up to 80\% can be achieved. Therefore, struvite production has large potential for reducing the phosphorus losses from livestock facilities. Considering only struvite formation from intensive cattle operations, reductions between 22\% and 36\% of the total phosphorus releases from the 	agricultural sector in the contiguous United States, including manure releases and fertilizer application, can be achieved.
%	 is performed in Chapter the deployment struvite production systems in the CAFOs of the US
	
	\item \textbf{Development of a decision-making support tool for the implementation of phosphorus recovery technologies at CAFOs.}
	
	
	
	\item \textbf{Design and assessment of incentive policies for the implementation of phosphorus recovery technologies at CAFOs, including the fair allocation of monetary resources in limited-budget scenarios.}
	
	\item \textbf{Potential integration of biogas production and nutrient recovery processes.} In Chapter \ref{ch:BiogasUpgrading}, a systematic study of different biogas upgrading to biomethane processes is performed to identify the optimal upgrading technology considering the particular characteristics of the biogas produced from livestock manure.  A mathematical optimization approach is used to compare the technologies technologies under evaluation using a non-linear programming (NLP) superstructure model. Pressure swing adsorption emerges as the most cost-effective technology for producing biomethane from raw biogas. Moreover, in Chapter \ref{ch:Policies} the integrations of biogas production and phosphorus recovery processes in real CAFOs is studied. The results show that The integration of phosphorus recovery technologies with anaerobic 	digestion and biogas upgrading processes does not result in any practical 	improvement in terms of economic performance unless incentives for 	phosphorus recovery are considered, since the revenues from electricity sales can not cover the investment and operating cost of these processes	given current market values.
\end{enumerate}







\section{Directions for future work}

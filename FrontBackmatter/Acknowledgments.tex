%*******************************************************
% Acknowledgments
%*******************************************************
\pdfbookmark[1]{Acknowledgments}{acknowledgments}

%\begin{flushright}{\slshape
%    We have seen that computer programming is an art, \\
%    because it applies accumulated knowledge to the world, \\
%    because it requires skill and ingenuity, and especially \\
%    because it produces objects of beauty.} \\ \medskip
%    --- \defcitealias{knuth:1974}{Donald E. Knuth}\citetalias{knuth:1974} \citep{knuth:1974}
%\end{flushright}



\bigskip

\begingroup
\let\clearpage\relax
\let\cleardoublepage\relax
\let\cleardoublepage\relax
\chapter*{Acknowledgments/Agradecimientos}
El desarrollo de esta tesis ha sido un proceso emocionante del que han formado parte muchas personas. Le estoy enormemente agradecido a todas ellas, a las que intentaré mencionar en las siguientes lineas. Sin embargo, si me olvido de alguna de ellas, le ruego que acepte mis mas sinceras disculpas, pero que tenga por seguro que le estoy tan agradecido como a las que aquí se mencionan.

Me gustaría comenzar por mi director de tesis, el Dr. Mariano Martín, por haber hecho posible desarollar este trabajo que no solamente se compone de los documentos que aqui se presentan, si no también de multiples experiencias, viajes y encuentros cuyo valor en inconmensurable. Igualmente, me gustaría agradecer al Dr. Gerardo Ruiz Mercado, de la U.S. Environmental Protection Agency, y al Prof. Victor M. Zavala, de la Universidad de Wisconsin-Madison, por haber contribuido en gran medida al desarrollo de este trabajo.

No puedo dejar de mencionar a los miembros (y sobretodo amigos) del grupo PSEM3 de la Universidad de Salamanca que me han acompañado durante este emocionante camino, Antonio, Manu, Borja, Lidia, Guillermo, y a nuestras nuevas (y no tan nuevas) incorporaciones, Carlos, Sofia y Elena. De manera especial, tambien me gustaria agradecer a Clara y Enrique, pues sus propios trabajos han contribuido a desarollar esta tesis. Y por su puesto, a todos aquellos que nos visitaron largo de estos años, y que espero encontrasen en Salamanca un lugar acogedor al que volver cuando lo deseen, César, Juan, Salvador, Gabriel, Valentina, Thalles y Javier.

An important piece of this thesis is the collaboration with Dr. Gerardo Ruiz Mercado, to whom I am very grateful for the opportunity of collaborate with him and the U.S. Environmental Protection Agency, and welcome me during my time Cincinnati. During this time I could meet some of the most kind people, who where a great support and became good friends, Marco, Gulizhaer, Jose, and Jessica. I also want to show my gratitude to Apoorva and Yicheng, who were graduate students at UW-Madison at the same time as I was, for the wonderful works I could collaborate with you, and your contributions to this thesis and to my Ph.D. journey. También quiero mostrarle mi agradecimiento a Rodrigo y Claudio, que me recibieron cálidamente en el CIPA durante mi viaje a Chile en los albores de esta tesis. 

Tan importante como los agradecimientos a aquellos que forman parte de la academia son aquellos dirigidos a aquellos que desde fuera apoyan esta empresa desde fuera, y que sean conscientes de ello o no, tambien dejan su huella en este trabajo, particularmente a Cristina G., Ana, Cristina M., Joaquín, Dani, y Blanca, y a Teresa y a Cristina. También he de mencionar a Luismi, Miky y Tom, con los que junto a Cris pude compartir una gran estancia en Bélgica en los primeros pasos de esta tesis, y a Camila y Gonzalo, con los que poco despues me encontré en Chile. 

No puede faltar un especial agradecimiento a mis padres y a mi hermano, que son los que me han permitdo estar aquí en el día de hoy. Finally, I feel the need to show a special appreciation to Evelyne, who appeared on this journey, and is part of it ever since.

\bigskip

\hspace*{\fill} Halifax (Canadá), 30 de octubre de 2021

\endgroup

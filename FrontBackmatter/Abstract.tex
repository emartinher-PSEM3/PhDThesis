%*******************************************************
% Abstract
%*******************************************************
%\renewcommand{\abstractname}{Abstract}
\pdfbookmark[1]{Abstract}{Abstract}
% \addcontentsline{toc}{chapter}{\tocEntry{Abstract}}
%\begingroup
%\let\clearpage\relax
%\let\cleardoublepage\relax
%\let\cleardoublepage\relax

\chapter*{Abstract}
%Short summary of the contents in English\dots a great guide by
%Kent Beck how to write good abstracts can be found here:
%\begin{center}
%\url{https://plg.uwaterloo.ca/~migod/research/beckOOPSLA.html}
%\end{center}

%startling sentence

%problem
Nutrient pollution of waterbodies is a major worldwide water quality problem. Excessive use and discharge of nutrients can lead to eutrophication of fresh and marine waters, resulting in environmental problems associated with algal blooms and hypoxia, public health issues related to the release of toxins, and freshwater scarcity. 
%Surveys reveal that eutrophication is a global problem, reporting that 54\% of lakes in Asia, 53\% in Europe, 48\% in North America, 41\% in South America, and 28\% in Africa are affected by eutrophication.

Agricultural activities are one of the main contributors to anthropogenic nutrient releases. Focusing on the livestock industry, the releases of nutrients (mainly phosphorus and nitrogen) result from the production of large amounts of organic waste. 
%While for animals on pasture, organic waste should not be a source of concern if stocking rates are not excessive, 
Particularly, the manure generated in the concentrated animal feeding oper­ations (CAFOs) is a considerable challenge due to the high rates and spatial concentration of the organic waste generated.
%The development of sustainable agricultural intensification techniques is not only a desirable but also a necessary measure to reduce the environmental impact of livestock industry while meeting the current and future food demand. 
The abatement of nutrient releases from livestock facilities is a step to address the environmental problem of nutrient pollution.

%solution
This thesis aims at the holistic assessment of waste treatment processes and management practices for the effective recovery of nutrients from livestock waste. We have performed techno-economic assessments of phosphorus and nitrogen recovery technologies for livestock facilities to determine the systems which implementation in CAFOs is more viable, as well as the potential integration of nutrient recovery technologies with biogas production systems. Based on the information obtained in these studies, a geospatial evaluation of the impact of phosphorus recovery by deploying phosphorus recovery systems at CAFOs in the watersheds of the United States has been carried out. After establishing the most suitable type of processes for phosphorus recovery, a decision-making support tool for the assessment and selection of phosphorus recovery technologies based on technical, economic, and environmental criteria of each CAFO has been developed. Finally, this tool has been used for the design and analysis of incentive policies to promote the implementation of phosphorus recovery processes at CAFOs, including the fair allocation of incentives in limited budget scenarios.

%defence
These studies are intended to contribute to the development and implementation of sustainable nutrient management strategies at livestock facilities, addressing one of the major water quality problems around the globe, and promoting the transition to a more sustainable paradigm for food production.


%related work

%\newpage
\cleardoublepage

\begin{otherlanguage}{spanish}
\pdfbookmark[1]{Resumen}{Resumen}
\chapter*{Resumen}
La contaminación por nutrientes de las masas de agua es uno de los principales problemas de calidad del agua en todo el mundo. El uso excesivo de nutrientes da lugar a la eutrofización de aguas dulces y marinas, resultando en problemas medioambientales relacionados con la proliferación de algas y la hipoxia de las aguas, así como problemas de salud pública y escasez de agua potable.
Las actividades agrícolas son uno de los principales contribuyentes a las emisiones antropogénicas de nutrientes. Si nos centramos en la industria ganadera, las liberaciones de nutrientes (principalmente fósforo y nitrógeno) son el resultado de la producción de grandes cantidades de residuos orgánicos. En particular, las deyecciones ganaderas provenientes de grandes instalaciones de ganadería intensiva son un reto de considerable importancia debido a las grandes cantidades de residuo generadas y su alta concentración espacial. 
%La reducción de las emisiones de nutrientes de las instalaciones ganaderas es un paso necesario para abordar el problema medioambiental de la contaminación por nutrientes.

Esta tesis tiene como objetivo llevar a cabo una evaluación holística de los procesos de tratamiento y los procedimientos de gestión de residuos para la recuperación efectiva de nutrientes de los residuos ganaderos. Se han realizado estudios tecno-económicos de las tecnologías de recuperación de fósforo y nitrógeno con el fin de determinar los sistemas cuya implementación en las instalaciones ganaderas es más viable, así como la posible integración de estos sistemas con procesos de producción de biogás. A partir de la información obtenida en estos estudios, se ha realizado una evaluación geoespacial del impacto de la recuperación de fósforo 
%llevada a cabo mediante la implementación de sistemas de recuperación de fósforo 
en instalaciones ganaderas en las diferentes cuencas hidrográficas de Estados Unidos. Tras establecer el tipo de procesos más adecuados para la recuperación de fósforo, se ha desarrollado una herramienta de soporte a la toma de decisiones para la selección de tecnologías comerciales de recuperación de fósforo acorde a criterios técnicos, económicos y ambientales de cada instalación ganadera. Por último, esta herramienta se ha utilizado para el diseño y análisis de políticas de incentivos para promover la implementación de estos procesos
%de recuperación de fósforo 
en instalaciones de ganadería intensiva, incluyendo la distribución equitativa de incentivos en escenarios de presupuesto limitado.

Se pretende que estos estudios contribuyan al desarrollo y aplicación de estrategias de gestión de los nutrientes liberados por la industria ganadera, abordando uno de los principales problemas globales relacionados con la calidad del agua, y promoviendo la transición hacía un paradigma para la producción de alimentos más sostenible.
\end{otherlanguage}

%\endgroup

\vfill
